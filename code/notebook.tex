
% Default to the notebook output style

    


% Inherit from the specified cell style.




    
\documentclass[11pt]{article}

    
    
    \usepackage[T1]{fontenc}
    % Nicer default font (+ math font) than Computer Modern for most use cases
    \usepackage{mathpazo}

    % Basic figure setup, for now with no caption control since it's done
    % automatically by Pandoc (which extracts ![](path) syntax from Markdown).
    \usepackage{graphicx}
    % We will generate all images so they have a width \maxwidth. This means
    % that they will get their normal width if they fit onto the page, but
    % are scaled down if they would overflow the margins.
    \makeatletter
    \def\maxwidth{\ifdim\Gin@nat@width>\linewidth\linewidth
    \else\Gin@nat@width\fi}
    \makeatother
    \let\Oldincludegraphics\includegraphics
    % Set max figure width to be 80% of text width, for now hardcoded.
    \renewcommand{\includegraphics}[1]{\Oldincludegraphics[width=.8\maxwidth]{#1}}
    % Ensure that by default, figures have no caption (until we provide a
    % proper Figure object with a Caption API and a way to capture that
    % in the conversion process - todo).
    \usepackage{caption}
    \DeclareCaptionLabelFormat{nolabel}{}
    \captionsetup{labelformat=nolabel}

    \usepackage{adjustbox} % Used to constrain images to a maximum size 
    \usepackage{xcolor} % Allow colors to be defined
    \usepackage{enumerate} % Needed for markdown enumerations to work
    \usepackage{geometry} % Used to adjust the document margins
    \usepackage{amsmath} % Equations
    \usepackage{amssymb} % Equations
    \usepackage{textcomp} % defines textquotesingle
    % Hack from http://tex.stackexchange.com/a/47451/13684:
    \AtBeginDocument{%
        \def\PYZsq{\textquotesingle}% Upright quotes in Pygmentized code
    }
    \usepackage{upquote} % Upright quotes for verbatim code
    \usepackage{eurosym} % defines \euro
    \usepackage[mathletters]{ucs} % Extended unicode (utf-8) support
    \usepackage[utf8x]{inputenc} % Allow utf-8 characters in the tex document
    \usepackage{fancyvrb} % verbatim replacement that allows latex
    \usepackage{grffile} % extends the file name processing of package graphics 
                         % to support a larger range 
    % The hyperref package gives us a pdf with properly built
    % internal navigation ('pdf bookmarks' for the table of contents,
    % internal cross-reference links, web links for URLs, etc.)
    \usepackage{hyperref}
    \usepackage{longtable} % longtable support required by pandoc >1.10
    \usepackage{booktabs}  % table support for pandoc > 1.12.2
    \usepackage[inline]{enumitem} % IRkernel/repr support (it uses the enumerate* environment)
    \usepackage[normalem]{ulem} % ulem is needed to support strikethroughs (\sout)
                                % normalem makes italics be italics, not underlines
    

    
    
    % Colors for the hyperref package
    \definecolor{urlcolor}{rgb}{0,.145,.698}
    \definecolor{linkcolor}{rgb}{.71,0.21,0.01}
    \definecolor{citecolor}{rgb}{.12,.54,.11}

    % ANSI colors
    \definecolor{ansi-black}{HTML}{3E424D}
    \definecolor{ansi-black-intense}{HTML}{282C36}
    \definecolor{ansi-red}{HTML}{E75C58}
    \definecolor{ansi-red-intense}{HTML}{B22B31}
    \definecolor{ansi-green}{HTML}{00A250}
    \definecolor{ansi-green-intense}{HTML}{007427}
    \definecolor{ansi-yellow}{HTML}{DDB62B}
    \definecolor{ansi-yellow-intense}{HTML}{B27D12}
    \definecolor{ansi-blue}{HTML}{208FFB}
    \definecolor{ansi-blue-intense}{HTML}{0065CA}
    \definecolor{ansi-magenta}{HTML}{D160C4}
    \definecolor{ansi-magenta-intense}{HTML}{A03196}
    \definecolor{ansi-cyan}{HTML}{60C6C8}
    \definecolor{ansi-cyan-intense}{HTML}{258F8F}
    \definecolor{ansi-white}{HTML}{C5C1B4}
    \definecolor{ansi-white-intense}{HTML}{A1A6B2}

    % commands and environments needed by pandoc snippets
    % extracted from the output of `pandoc -s`
    \providecommand{\tightlist}{%
      \setlength{\itemsep}{0pt}\setlength{\parskip}{0pt}}
    \DefineVerbatimEnvironment{Highlighting}{Verbatim}{commandchars=\\\{\}}
    % Add ',fontsize=\small' for more characters per line
    \newenvironment{Shaded}{}{}
    \newcommand{\KeywordTok}[1]{\textcolor[rgb]{0.00,0.44,0.13}{\textbf{{#1}}}}
    \newcommand{\DataTypeTok}[1]{\textcolor[rgb]{0.56,0.13,0.00}{{#1}}}
    \newcommand{\DecValTok}[1]{\textcolor[rgb]{0.25,0.63,0.44}{{#1}}}
    \newcommand{\BaseNTok}[1]{\textcolor[rgb]{0.25,0.63,0.44}{{#1}}}
    \newcommand{\FloatTok}[1]{\textcolor[rgb]{0.25,0.63,0.44}{{#1}}}
    \newcommand{\CharTok}[1]{\textcolor[rgb]{0.25,0.44,0.63}{{#1}}}
    \newcommand{\StringTok}[1]{\textcolor[rgb]{0.25,0.44,0.63}{{#1}}}
    \newcommand{\CommentTok}[1]{\textcolor[rgb]{0.38,0.63,0.69}{\textit{{#1}}}}
    \newcommand{\OtherTok}[1]{\textcolor[rgb]{0.00,0.44,0.13}{{#1}}}
    \newcommand{\AlertTok}[1]{\textcolor[rgb]{1.00,0.00,0.00}{\textbf{{#1}}}}
    \newcommand{\FunctionTok}[1]{\textcolor[rgb]{0.02,0.16,0.49}{{#1}}}
    \newcommand{\RegionMarkerTok}[1]{{#1}}
    \newcommand{\ErrorTok}[1]{\textcolor[rgb]{1.00,0.00,0.00}{\textbf{{#1}}}}
    \newcommand{\NormalTok}[1]{{#1}}
    
    % Additional commands for more recent versions of Pandoc
    \newcommand{\ConstantTok}[1]{\textcolor[rgb]{0.53,0.00,0.00}{{#1}}}
    \newcommand{\SpecialCharTok}[1]{\textcolor[rgb]{0.25,0.44,0.63}{{#1}}}
    \newcommand{\VerbatimStringTok}[1]{\textcolor[rgb]{0.25,0.44,0.63}{{#1}}}
    \newcommand{\SpecialStringTok}[1]{\textcolor[rgb]{0.73,0.40,0.53}{{#1}}}
    \newcommand{\ImportTok}[1]{{#1}}
    \newcommand{\DocumentationTok}[1]{\textcolor[rgb]{0.73,0.13,0.13}{\textit{{#1}}}}
    \newcommand{\AnnotationTok}[1]{\textcolor[rgb]{0.38,0.63,0.69}{\textbf{\textit{{#1}}}}}
    \newcommand{\CommentVarTok}[1]{\textcolor[rgb]{0.38,0.63,0.69}{\textbf{\textit{{#1}}}}}
    \newcommand{\VariableTok}[1]{\textcolor[rgb]{0.10,0.09,0.49}{{#1}}}
    \newcommand{\ControlFlowTok}[1]{\textcolor[rgb]{0.00,0.44,0.13}{\textbf{{#1}}}}
    \newcommand{\OperatorTok}[1]{\textcolor[rgb]{0.40,0.40,0.40}{{#1}}}
    \newcommand{\BuiltInTok}[1]{{#1}}
    \newcommand{\ExtensionTok}[1]{{#1}}
    \newcommand{\PreprocessorTok}[1]{\textcolor[rgb]{0.74,0.48,0.00}{{#1}}}
    \newcommand{\AttributeTok}[1]{\textcolor[rgb]{0.49,0.56,0.16}{{#1}}}
    \newcommand{\InformationTok}[1]{\textcolor[rgb]{0.38,0.63,0.69}{\textbf{\textit{{#1}}}}}
    \newcommand{\WarningTok}[1]{\textcolor[rgb]{0.38,0.63,0.69}{\textbf{\textit{{#1}}}}}
    
    
    % Define a nice break command that doesn't care if a line doesn't already
    % exist.
    \def\br{\hspace*{\fill} \\* }
    % Math Jax compatability definitions
    \def\gt{>}
    \def\lt{<}
    % Document parameters
    \title{EKG Circuit Optimization Final}
    
    
    

    % Pygments definitions
    
\makeatletter
\def\PY@reset{\let\PY@it=\relax \let\PY@bf=\relax%
    \let\PY@ul=\relax \let\PY@tc=\relax%
    \let\PY@bc=\relax \let\PY@ff=\relax}
\def\PY@tok#1{\csname PY@tok@#1\endcsname}
\def\PY@toks#1+{\ifx\relax#1\empty\else%
    \PY@tok{#1}\expandafter\PY@toks\fi}
\def\PY@do#1{\PY@bc{\PY@tc{\PY@ul{%
    \PY@it{\PY@bf{\PY@ff{#1}}}}}}}
\def\PY#1#2{\PY@reset\PY@toks#1+\relax+\PY@do{#2}}

\expandafter\def\csname PY@tok@w\endcsname{\def\PY@tc##1{\textcolor[rgb]{0.73,0.73,0.73}{##1}}}
\expandafter\def\csname PY@tok@c\endcsname{\let\PY@it=\textit\def\PY@tc##1{\textcolor[rgb]{0.25,0.50,0.50}{##1}}}
\expandafter\def\csname PY@tok@cp\endcsname{\def\PY@tc##1{\textcolor[rgb]{0.74,0.48,0.00}{##1}}}
\expandafter\def\csname PY@tok@k\endcsname{\let\PY@bf=\textbf\def\PY@tc##1{\textcolor[rgb]{0.00,0.50,0.00}{##1}}}
\expandafter\def\csname PY@tok@kp\endcsname{\def\PY@tc##1{\textcolor[rgb]{0.00,0.50,0.00}{##1}}}
\expandafter\def\csname PY@tok@kt\endcsname{\def\PY@tc##1{\textcolor[rgb]{0.69,0.00,0.25}{##1}}}
\expandafter\def\csname PY@tok@o\endcsname{\def\PY@tc##1{\textcolor[rgb]{0.40,0.40,0.40}{##1}}}
\expandafter\def\csname PY@tok@ow\endcsname{\let\PY@bf=\textbf\def\PY@tc##1{\textcolor[rgb]{0.67,0.13,1.00}{##1}}}
\expandafter\def\csname PY@tok@nb\endcsname{\def\PY@tc##1{\textcolor[rgb]{0.00,0.50,0.00}{##1}}}
\expandafter\def\csname PY@tok@nf\endcsname{\def\PY@tc##1{\textcolor[rgb]{0.00,0.00,1.00}{##1}}}
\expandafter\def\csname PY@tok@nc\endcsname{\let\PY@bf=\textbf\def\PY@tc##1{\textcolor[rgb]{0.00,0.00,1.00}{##1}}}
\expandafter\def\csname PY@tok@nn\endcsname{\let\PY@bf=\textbf\def\PY@tc##1{\textcolor[rgb]{0.00,0.00,1.00}{##1}}}
\expandafter\def\csname PY@tok@ne\endcsname{\let\PY@bf=\textbf\def\PY@tc##1{\textcolor[rgb]{0.82,0.25,0.23}{##1}}}
\expandafter\def\csname PY@tok@nv\endcsname{\def\PY@tc##1{\textcolor[rgb]{0.10,0.09,0.49}{##1}}}
\expandafter\def\csname PY@tok@no\endcsname{\def\PY@tc##1{\textcolor[rgb]{0.53,0.00,0.00}{##1}}}
\expandafter\def\csname PY@tok@nl\endcsname{\def\PY@tc##1{\textcolor[rgb]{0.63,0.63,0.00}{##1}}}
\expandafter\def\csname PY@tok@ni\endcsname{\let\PY@bf=\textbf\def\PY@tc##1{\textcolor[rgb]{0.60,0.60,0.60}{##1}}}
\expandafter\def\csname PY@tok@na\endcsname{\def\PY@tc##1{\textcolor[rgb]{0.49,0.56,0.16}{##1}}}
\expandafter\def\csname PY@tok@nt\endcsname{\let\PY@bf=\textbf\def\PY@tc##1{\textcolor[rgb]{0.00,0.50,0.00}{##1}}}
\expandafter\def\csname PY@tok@nd\endcsname{\def\PY@tc##1{\textcolor[rgb]{0.67,0.13,1.00}{##1}}}
\expandafter\def\csname PY@tok@s\endcsname{\def\PY@tc##1{\textcolor[rgb]{0.73,0.13,0.13}{##1}}}
\expandafter\def\csname PY@tok@sd\endcsname{\let\PY@it=\textit\def\PY@tc##1{\textcolor[rgb]{0.73,0.13,0.13}{##1}}}
\expandafter\def\csname PY@tok@si\endcsname{\let\PY@bf=\textbf\def\PY@tc##1{\textcolor[rgb]{0.73,0.40,0.53}{##1}}}
\expandafter\def\csname PY@tok@se\endcsname{\let\PY@bf=\textbf\def\PY@tc##1{\textcolor[rgb]{0.73,0.40,0.13}{##1}}}
\expandafter\def\csname PY@tok@sr\endcsname{\def\PY@tc##1{\textcolor[rgb]{0.73,0.40,0.53}{##1}}}
\expandafter\def\csname PY@tok@ss\endcsname{\def\PY@tc##1{\textcolor[rgb]{0.10,0.09,0.49}{##1}}}
\expandafter\def\csname PY@tok@sx\endcsname{\def\PY@tc##1{\textcolor[rgb]{0.00,0.50,0.00}{##1}}}
\expandafter\def\csname PY@tok@m\endcsname{\def\PY@tc##1{\textcolor[rgb]{0.40,0.40,0.40}{##1}}}
\expandafter\def\csname PY@tok@gh\endcsname{\let\PY@bf=\textbf\def\PY@tc##1{\textcolor[rgb]{0.00,0.00,0.50}{##1}}}
\expandafter\def\csname PY@tok@gu\endcsname{\let\PY@bf=\textbf\def\PY@tc##1{\textcolor[rgb]{0.50,0.00,0.50}{##1}}}
\expandafter\def\csname PY@tok@gd\endcsname{\def\PY@tc##1{\textcolor[rgb]{0.63,0.00,0.00}{##1}}}
\expandafter\def\csname PY@tok@gi\endcsname{\def\PY@tc##1{\textcolor[rgb]{0.00,0.63,0.00}{##1}}}
\expandafter\def\csname PY@tok@gr\endcsname{\def\PY@tc##1{\textcolor[rgb]{1.00,0.00,0.00}{##1}}}
\expandafter\def\csname PY@tok@ge\endcsname{\let\PY@it=\textit}
\expandafter\def\csname PY@tok@gs\endcsname{\let\PY@bf=\textbf}
\expandafter\def\csname PY@tok@gp\endcsname{\let\PY@bf=\textbf\def\PY@tc##1{\textcolor[rgb]{0.00,0.00,0.50}{##1}}}
\expandafter\def\csname PY@tok@go\endcsname{\def\PY@tc##1{\textcolor[rgb]{0.53,0.53,0.53}{##1}}}
\expandafter\def\csname PY@tok@gt\endcsname{\def\PY@tc##1{\textcolor[rgb]{0.00,0.27,0.87}{##1}}}
\expandafter\def\csname PY@tok@err\endcsname{\def\PY@bc##1{\setlength{\fboxsep}{0pt}\fcolorbox[rgb]{1.00,0.00,0.00}{1,1,1}{\strut ##1}}}
\expandafter\def\csname PY@tok@kc\endcsname{\let\PY@bf=\textbf\def\PY@tc##1{\textcolor[rgb]{0.00,0.50,0.00}{##1}}}
\expandafter\def\csname PY@tok@kd\endcsname{\let\PY@bf=\textbf\def\PY@tc##1{\textcolor[rgb]{0.00,0.50,0.00}{##1}}}
\expandafter\def\csname PY@tok@kn\endcsname{\let\PY@bf=\textbf\def\PY@tc##1{\textcolor[rgb]{0.00,0.50,0.00}{##1}}}
\expandafter\def\csname PY@tok@kr\endcsname{\let\PY@bf=\textbf\def\PY@tc##1{\textcolor[rgb]{0.00,0.50,0.00}{##1}}}
\expandafter\def\csname PY@tok@bp\endcsname{\def\PY@tc##1{\textcolor[rgb]{0.00,0.50,0.00}{##1}}}
\expandafter\def\csname PY@tok@fm\endcsname{\def\PY@tc##1{\textcolor[rgb]{0.00,0.00,1.00}{##1}}}
\expandafter\def\csname PY@tok@vc\endcsname{\def\PY@tc##1{\textcolor[rgb]{0.10,0.09,0.49}{##1}}}
\expandafter\def\csname PY@tok@vg\endcsname{\def\PY@tc##1{\textcolor[rgb]{0.10,0.09,0.49}{##1}}}
\expandafter\def\csname PY@tok@vi\endcsname{\def\PY@tc##1{\textcolor[rgb]{0.10,0.09,0.49}{##1}}}
\expandafter\def\csname PY@tok@vm\endcsname{\def\PY@tc##1{\textcolor[rgb]{0.10,0.09,0.49}{##1}}}
\expandafter\def\csname PY@tok@sa\endcsname{\def\PY@tc##1{\textcolor[rgb]{0.73,0.13,0.13}{##1}}}
\expandafter\def\csname PY@tok@sb\endcsname{\def\PY@tc##1{\textcolor[rgb]{0.73,0.13,0.13}{##1}}}
\expandafter\def\csname PY@tok@sc\endcsname{\def\PY@tc##1{\textcolor[rgb]{0.73,0.13,0.13}{##1}}}
\expandafter\def\csname PY@tok@dl\endcsname{\def\PY@tc##1{\textcolor[rgb]{0.73,0.13,0.13}{##1}}}
\expandafter\def\csname PY@tok@s2\endcsname{\def\PY@tc##1{\textcolor[rgb]{0.73,0.13,0.13}{##1}}}
\expandafter\def\csname PY@tok@sh\endcsname{\def\PY@tc##1{\textcolor[rgb]{0.73,0.13,0.13}{##1}}}
\expandafter\def\csname PY@tok@s1\endcsname{\def\PY@tc##1{\textcolor[rgb]{0.73,0.13,0.13}{##1}}}
\expandafter\def\csname PY@tok@mb\endcsname{\def\PY@tc##1{\textcolor[rgb]{0.40,0.40,0.40}{##1}}}
\expandafter\def\csname PY@tok@mf\endcsname{\def\PY@tc##1{\textcolor[rgb]{0.40,0.40,0.40}{##1}}}
\expandafter\def\csname PY@tok@mh\endcsname{\def\PY@tc##1{\textcolor[rgb]{0.40,0.40,0.40}{##1}}}
\expandafter\def\csname PY@tok@mi\endcsname{\def\PY@tc##1{\textcolor[rgb]{0.40,0.40,0.40}{##1}}}
\expandafter\def\csname PY@tok@il\endcsname{\def\PY@tc##1{\textcolor[rgb]{0.40,0.40,0.40}{##1}}}
\expandafter\def\csname PY@tok@mo\endcsname{\def\PY@tc##1{\textcolor[rgb]{0.40,0.40,0.40}{##1}}}
\expandafter\def\csname PY@tok@ch\endcsname{\let\PY@it=\textit\def\PY@tc##1{\textcolor[rgb]{0.25,0.50,0.50}{##1}}}
\expandafter\def\csname PY@tok@cm\endcsname{\let\PY@it=\textit\def\PY@tc##1{\textcolor[rgb]{0.25,0.50,0.50}{##1}}}
\expandafter\def\csname PY@tok@cpf\endcsname{\let\PY@it=\textit\def\PY@tc##1{\textcolor[rgb]{0.25,0.50,0.50}{##1}}}
\expandafter\def\csname PY@tok@c1\endcsname{\let\PY@it=\textit\def\PY@tc##1{\textcolor[rgb]{0.25,0.50,0.50}{##1}}}
\expandafter\def\csname PY@tok@cs\endcsname{\let\PY@it=\textit\def\PY@tc##1{\textcolor[rgb]{0.25,0.50,0.50}{##1}}}

\def\PYZbs{\char`\\}
\def\PYZus{\char`\_}
\def\PYZob{\char`\{}
\def\PYZcb{\char`\}}
\def\PYZca{\char`\^}
\def\PYZam{\char`\&}
\def\PYZlt{\char`\<}
\def\PYZgt{\char`\>}
\def\PYZsh{\char`\#}
\def\PYZpc{\char`\%}
\def\PYZdl{\char`\$}
\def\PYZhy{\char`\-}
\def\PYZsq{\char`\'}
\def\PYZdq{\char`\"}
\def\PYZti{\char`\~}
% for compatibility with earlier versions
\def\PYZat{@}
\def\PYZlb{[}
\def\PYZrb{]}
\makeatother


    % Exact colors from NB
    \definecolor{incolor}{rgb}{0.0, 0.0, 0.5}
    \definecolor{outcolor}{rgb}{0.545, 0.0, 0.0}



    
    % Prevent overflowing lines due to hard-to-break entities
    \sloppy 
    % Setup hyperref package
    \hypersetup{
      breaklinks=true,  % so long urls are correctly broken across lines
      colorlinks=true,
      urlcolor=urlcolor,
      linkcolor=linkcolor,
      citecolor=citecolor,
      }
    % Slightly bigger margins than the latex defaults
    
    \geometry{verbose,tmargin=1in,bmargin=1in,lmargin=1in,rmargin=1in}
    
    

    \begin{document}
    
    
    \maketitle
    
    

    
    \section{Optimizing EKG Circuit}\label{optimizing-ekg-circuit}

    Created by Thomas Jagielski and Arwen Sadler

    \begin{Verbatim}[commandchars=\\\{\}]
{\color{incolor}In [{\color{incolor}1}]:} \PY{c+c1}{\PYZsh{} Configure Jupyter so figures appear in the notebook}
        \PY{o}{\PYZpc{}}\PY{k}{matplotlib} inline
        
        \PY{c+c1}{\PYZsh{} Configure Jupyter to display the assigned value after an assignment}
        \PY{o}{\PYZpc{}}\PY{k}{config} InteractiveShell.ast\PYZus{}node\PYZus{}interactivity=\PYZsq{}last\PYZus{}expr\PYZus{}or\PYZus{}assign\PYZsq{}
        
        \PY{c+c1}{\PYZsh{} import functions from the modsim.py module}
        \PY{k+kn}{from} \PY{n+nn}{modsim} \PY{k}{import} \PY{o}{*}
\end{Verbatim}


    \subsubsection{Question:}\label{question}

    For this project we were interested in modeling the EKG circuit we built
in Introduction to Sensors, Instrumentation and Measurements (ISIM). The
question we addressed was, how does this circuit respond with changes in
RC values?

    \paragraph{Diagram of the Circuit
Modeled:}\label{diagram-of-the-circuit-modeled}

    *Also referenced at end in PDF Verison.

    \begin{figure}
\centering
\includegraphics{attachment:EKG\%20Amplifier\%201.png}
\caption{EKG\%20Amplifier\%201.png}
\end{figure}

    \begin{figure}
\centering
\includegraphics{attachment:EKG\%20Amplifier\%202.PNG}
\caption{EKG\%20Amplifier\%202.PNG}
\end{figure}

    \subsubsection{Model 1:}\label{model-1}

    \paragraph{Assumptions:}\label{assumptions}

    \begin{itemize}
\tightlist
\item
  We model the input heart rate is a cosine wave
\item
  There is no electrical interference
\item
  We do not include the input impedance of the analog discovery, which
  is used to measure output voltage
\end{itemize}

    Below we set the input signal frequency similar to the frequency of the
average heart rate. This data is taken from the Cleveland Clinic,
https://my.clevelandclinic.org/health/diagnostics/17402-pulse-\/-heart-rate,
which states the average heart rate of an adult to be 60 - 100 beats per
minute. Converted to frequency, in Hz, this is 1 Hz.

\[\frac{60 \textrm{ beats}}{1 \textrm{ minute}} * \frac{1 \textrm{ minute}}{60 \textrm{ seconds}} = 1 \textrm{ beat per second}  \textrm {, so}\]

    \[ 1 \textrm{ beat per second} = 1 \textrm{ Hz}\]

    \begin{Verbatim}[commandchars=\\\{\}]
{\color{incolor}In [{\color{incolor}2}]:} \PY{c+c1}{\PYZsh{}Change the input signal frequnecy}
        \PY{n}{input\PYZus{}freq} \PY{o}{=} \PY{n}{Params}\PY{p}{(}
            \PY{n}{f} \PY{o}{=} \PY{l+m+mi}{1}
        \PY{p}{)}
\end{Verbatim}


\begin{Verbatim}[commandchars=\\\{\}]
{\color{outcolor}Out[{\color{outcolor}2}]:} f    1
        dtype: int64
\end{Verbatim}
            
    We then initialize the first filter's parameters.

    \begin{Verbatim}[commandchars=\\\{\}]
{\color{incolor}In [{\color{incolor}3}]:} \PY{n}{params1} \PY{o}{=} \PY{n}{Params}\PY{p}{(}
            \PY{n}{R} \PY{o}{=} \PY{l+m+mf}{4.9e3}\PY{p}{,}   \PY{c+c1}{\PYZsh{} ohm}
            \PY{n}{C} \PY{o}{=} \PY{l+m+mf}{1e\PYZhy{}6}\PY{p}{,}  \PY{c+c1}{\PYZsh{} farad}
            \PY{n}{A} \PY{o}{=} \PY{l+m+mi}{5}\PY{p}{,}      \PY{c+c1}{\PYZsh{} volt}
            \PY{n}{f} \PY{o}{=} \PY{n}{input\PYZus{}freq}\PY{o}{.}\PY{n}{f}\PY{p}{,}   \PY{c+c1}{\PYZsh{} Hz}
            \PY{n}{vin} \PY{o}{=} \PY{l+m+mi}{0}
        \PY{p}{)}
\end{Verbatim}


\begin{Verbatim}[commandchars=\\\{\}]
{\color{outcolor}Out[{\color{outcolor}3}]:} R      4900.000000
        C         0.000001
        A         5.000000
        f         1.000000
        vin       0.000000
        dtype: float64
\end{Verbatim}
            
    We then define a function to make a system object with the input
parameters. This allows us to make system objects by creating new
parameter objects.

    \begin{Verbatim}[commandchars=\\\{\}]
{\color{incolor}In [{\color{incolor}4}]:} \PY{k}{def} \PY{n+nf}{make\PYZus{}system}\PY{p}{(}\PY{n}{params}\PY{p}{)}\PY{p}{:}
            \PY{l+s+sd}{\PYZdq{}\PYZdq{}\PYZdq{}Makes a System object for the given conditions.}
        \PY{l+s+sd}{    }
        \PY{l+s+sd}{    params: Params object}
        \PY{l+s+sd}{    }
        \PY{l+s+sd}{    returns: System object}
        \PY{l+s+sd}{    \PYZdq{}\PYZdq{}\PYZdq{}}
            \PY{n}{unpack}\PY{p}{(}\PY{n}{params}\PY{p}{)}
            
            \PY{n}{init} \PY{o}{=} \PY{n}{State}\PY{p}{(}\PY{n}{V\PYZus{}out} \PY{o}{=} \PY{l+m+mi}{0}\PY{p}{)}
            \PY{c+c1}{\PYZsh{} Omega is eaual to angular frequency}
            \PY{n}{omega} \PY{o}{=} \PY{l+m+mi}{2} \PY{o}{*} \PY{n}{np}\PY{o}{.}\PY{n}{pi} \PY{o}{*} \PY{n}{input\PYZus{}freq}\PY{o}{.}\PY{n}{f}
            \PY{c+c1}{\PYZsh{} Tau is equal to the RC Value}
            \PY{n}{tau} \PY{o}{=} \PY{n}{R} \PY{o}{*} \PY{n}{C}
            \PY{c+c1}{\PYZsh{} The cutoff frequnecy shows where the filters take effect}
            \PY{n}{cutoff} \PY{o}{=} \PY{l+m+mi}{1} \PY{o}{/} \PY{n}{R} \PY{o}{/} \PY{n}{C} \PY{o}{/} \PY{l+m+mi}{2} \PY{o}{/} \PY{n}{np}\PY{o}{.}\PY{n}{pi}
            \PY{n}{t\PYZus{}end} \PY{o}{=} \PY{l+m+mi}{4} \PY{o}{/} \PY{n}{input\PYZus{}freq}\PY{o}{.}\PY{n}{f}
            \PY{n}{ts} \PY{o}{=} \PY{n}{linspace}\PY{p}{(}\PY{l+m+mi}{0}\PY{p}{,} \PY{n}{t\PYZus{}end}\PY{p}{,} \PY{l+m+mi}{401}\PY{p}{)}
            
            \PY{k}{return} \PY{n}{System}\PY{p}{(}\PY{n}{R}\PY{o}{=}\PY{n}{R}\PY{p}{,} \PY{n}{C}\PY{o}{=}\PY{n}{C}\PY{p}{,} \PY{n}{A}\PY{o}{=}\PY{n}{A}\PY{p}{,} \PY{n}{f}\PY{o}{=}\PY{n}{input\PYZus{}freq}\PY{o}{.}\PY{n}{f}\PY{p}{,}
                          \PY{n}{init}\PY{o}{=}\PY{n}{init}\PY{p}{,} \PY{n}{t\PYZus{}end}\PY{o}{=}\PY{n}{t\PYZus{}end}\PY{p}{,} \PY{n}{ts}\PY{o}{=}\PY{n}{ts}\PY{p}{,}
                          \PY{n}{omega}\PY{o}{=}\PY{n}{omega}\PY{p}{,} \PY{n}{tau}\PY{o}{=}\PY{n}{tau}\PY{p}{,} \PY{n}{cutoff}\PY{o}{=}\PY{n}{cutoff}\PY{p}{,} \PY{n}{vin}\PY{o}{=}\PY{n}{vin}\PY{p}{)}
\end{Verbatim}


    \begin{Verbatim}[commandchars=\\\{\}]
{\color{incolor}In [{\color{incolor}5}]:} \PY{c+c1}{\PYZsh{} Use params1 to make a system object}
        \PY{n}{system1} \PY{o}{=} \PY{n}{make\PYZus{}system}\PY{p}{(}\PY{n}{params1}\PY{p}{)}
\end{Verbatim}


\begin{Verbatim}[commandchars=\\\{\}]
{\color{outcolor}Out[{\color{outcolor}5}]:} R                                                      4900
        C                                                     1e-06
        A                                                         5
        f                                                         1
        init                                V\_out    0
        dtype: int64
        t\_end                                                     4
        ts        [0.0, 0.01, 0.02, 0.03, 0.04, 0.05, 0.06, 0.07{\ldots}
        omega                                               6.28319
        tau                                                  0.0049
        cutoff                                              32.4806
        vin                                                       0
        dtype: object
\end{Verbatim}
            
    We plot the input signal on a chart using the system1 parameters.

    \begin{Verbatim}[commandchars=\\\{\}]
{\color{incolor}In [{\color{incolor}6}]:} \PY{n}{x} \PY{o}{=} \PY{n}{linspace}\PY{p}{(}\PY{l+m+mi}{0}\PY{p}{,} \PY{n}{system1}\PY{o}{.}\PY{n}{t\PYZus{}end} \PY{o}{*} \PY{l+m+mi}{1000}\PY{p}{,} \PY{l+m+mi}{401}\PY{p}{)}
        \PY{n}{v} \PY{o}{=} \PY{n}{system1}\PY{o}{.}\PY{n}{A} \PY{o}{*} \PY{n}{np}\PY{o}{.}\PY{n}{cos}\PY{p}{(}\PY{l+m+mi}{2} \PY{o}{*} \PY{n}{pi} \PY{o}{*} \PY{p}{(}\PY{n}{system1}\PY{o}{.}\PY{n}{f} \PY{o}{/} \PY{l+m+mi}{1000}\PY{p}{)} \PY{o}{*} \PY{n}{x}\PY{p}{)}
        \PY{n}{plt}\PY{o}{.}\PY{n}{plot}\PY{p}{(}\PY{n}{x}\PY{p}{,} \PY{n}{v}\PY{p}{)}
        \PY{n}{decorate}\PY{p}{(}\PY{n}{xlabel}\PY{o}{=}\PY{l+s+s1}{\PYZsq{}}\PY{l+s+s1}{Time (ms)}\PY{l+s+s1}{\PYZsq{}}\PY{p}{,}
                     \PY{n}{ylabel}\PY{o}{=}\PY{l+s+s1}{\PYZsq{}}\PY{l+s+s1}{\PYZdl{}V\PYZus{}}\PY{l+s+si}{\PYZob{}out\PYZcb{}}\PY{l+s+s1}{\PYZdl{} (volt)}\PY{l+s+s1}{\PYZsq{}}\PY{p}{,}
                     \PY{n}{title}\PY{o}{=}\PY{l+s+s1}{\PYZsq{}}\PY{l+s+s1}{Input Signal}\PY{l+s+s1}{\PYZsq{}}\PY{p}{,}
                     \PY{n}{legend}\PY{o}{=}\PY{k+kc}{False}\PY{p}{)}
\end{Verbatim}


    \begin{center}
    \adjustimage{max size={0.9\linewidth}{0.9\paperheight}}{output_21_0.png}
    \end{center}
    { \hspace*{\fill} \\}
    
    We define a slope function to express the change in voltage across the
filter. This models the effect of the filter based on RC values and the
frequency of the input signal. We run this function with the
run\_ode\_solver.

    \begin{Verbatim}[commandchars=\\\{\}]
{\color{incolor}In [{\color{incolor}7}]:} \PY{k}{def} \PY{n+nf}{slope\PYZus{}func\PYZus{}init}\PY{p}{(}\PY{n}{state}\PY{p}{,} \PY{n}{t}\PY{p}{,} \PY{n}{system}\PY{p}{)}\PY{p}{:}
            \PY{l+s+sd}{\PYZdq{}\PYZdq{}\PYZdq{}Makes a slope function to update the state.}
        \PY{l+s+sd}{    }
        \PY{l+s+sd}{    state: State(V\PYZus{}out)}
        \PY{l+s+sd}{    t: time}
        \PY{l+s+sd}{    system: System Object}
        \PY{l+s+sd}{    }
        \PY{l+s+sd}{    returns: State(V\PYZus{}out)\PYZdq{}\PYZdq{}\PYZdq{}}
        
            \PY{n}{vout} \PY{o}{=} \PY{n}{state}
            
            \PY{n}{unpack}\PY{p}{(}\PY{n}{system}\PY{p}{)}
            
            \PY{n}{vin} \PY{o}{=} \PY{n}{A} \PY{o}{*} \PY{n}{np}\PY{o}{.}\PY{n}{cos}\PY{p}{(}\PY{l+m+mi}{2} \PY{o}{*} \PY{n}{pi} \PY{o}{*} \PY{n}{f} \PY{o}{*} \PY{n}{t}\PY{p}{)}
            
            \PY{n}{dvoutdt} \PY{o}{=} \PY{p}{(}\PY{n}{vin} \PY{o}{\PYZhy{}} \PY{n}{vout}\PY{p}{)} \PY{o}{/} \PY{p}{(}\PY{n}{R} \PY{o}{*} \PY{n}{C}\PY{p}{)}
            
            \PY{k}{return} \PY{n}{dvoutdt}
\end{Verbatim}


    \begin{Verbatim}[commandchars=\\\{\}]
{\color{incolor}In [{\color{incolor}8}]:} \PY{c+c1}{\PYZsh{} Run system1 through slope\PYZus{}func\PYZus{}init using the run\PYZus{}ode\PYZus{}sovler}
        \PY{n}{results1}\PY{p}{,} \PY{n}{details1} \PY{o}{=} \PY{n}{run\PYZus{}ode\PYZus{}solver}\PY{p}{(}\PY{n}{system1}\PY{p}{,} 
                                            \PY{n}{slope\PYZus{}func\PYZus{}init}\PY{p}{,} 
                                            \PY{n}{t\PYZus{}eval}\PY{o}{=}\PY{n}{system1}\PY{o}{.}\PY{n}{ts}\PY{p}{)}
\end{Verbatim}


    \begin{Verbatim}[commandchars=\\\{\}]
{\color{incolor}In [{\color{incolor}9}]:} \PY{k}{def} \PY{n+nf}{plot\PYZus{}results}\PY{p}{(}\PY{n}{results}\PY{p}{)}\PY{p}{:}
            \PY{l+s+sd}{\PYZdq{}\PYZdq{}\PYZdq{}Makes function to plot the results given in a DataFrame.}
        \PY{l+s+sd}{    }
        \PY{l+s+sd}{    results: DataFrame of the results of }
        \PY{l+s+sd}{             run\PYZus{}ode\PYZus{}solver on a slope function}
        \PY{l+s+sd}{    }
        \PY{l+s+sd}{    returns: a plot of the results\PYZdq{}\PYZdq{}\PYZdq{}}
            
            \PY{n}{xs} \PY{o}{=} \PY{n}{results}\PY{o}{.}\PY{n}{V\PYZus{}out}\PY{o}{.}\PY{n}{index}
            \PY{n}{ys} \PY{o}{=} \PY{n}{results}\PY{o}{.}\PY{n}{V\PYZus{}out}\PY{o}{.}\PY{n}{values}
        
            \PY{n}{t\PYZus{}end} \PY{o}{=} \PY{n}{get\PYZus{}last\PYZus{}label}\PY{p}{(}\PY{n}{results}\PY{p}{)}
            \PY{k}{if} \PY{n}{t\PYZus{}end} \PY{o}{\PYZlt{}} \PY{l+m+mi}{10}\PY{p}{:}
                \PY{n}{xs} \PY{o}{*}\PY{o}{=} \PY{l+m+mi}{1000}
                \PY{n}{xlabel} \PY{o}{=} \PY{l+s+s1}{\PYZsq{}}\PY{l+s+s1}{Time (ms)}\PY{l+s+s1}{\PYZsq{}}
            \PY{k}{else}\PY{p}{:}
                \PY{n}{xlabel} \PY{o}{=} \PY{l+s+s1}{\PYZsq{}}\PY{l+s+s1}{Time (s)}\PY{l+s+s1}{\PYZsq{}}
                
            \PY{n}{plot}\PY{p}{(}\PY{n}{xs}\PY{p}{,} \PY{n}{ys}\PY{p}{)}
            \PY{n}{decorate}\PY{p}{(}\PY{n}{xlabel}\PY{o}{=}\PY{n}{xlabel}\PY{p}{,}
                     \PY{n}{ylabel}\PY{o}{=}\PY{l+s+s1}{\PYZsq{}}\PY{l+s+s1}{\PYZdl{}V\PYZus{}}\PY{l+s+si}{\PYZob{}out\PYZcb{}}\PY{l+s+s1}{\PYZdl{} (volt)}\PY{l+s+s1}{\PYZsq{}}\PY{p}{,}
                     \PY{n}{legend}\PY{o}{=}\PY{k+kc}{False}\PY{p}{)}
\end{Verbatim}


    We then set the parameters of the second filter and make a system of
these values.

    \begin{Verbatim}[commandchars=\\\{\}]
{\color{incolor}In [{\color{incolor}10}]:} \PY{n}{params2} \PY{o}{=} \PY{n}{Params}\PY{p}{(}
             \PY{n}{R} \PY{o}{=} \PY{l+m+mf}{100e3}\PY{p}{,}   \PY{c+c1}{\PYZsh{} ohm}
             \PY{n}{C} \PY{o}{=} \PY{l+m+mf}{1e\PYZhy{}6}\PY{p}{,}  \PY{c+c1}{\PYZsh{} farad}
             \PY{n}{vin} \PY{o}{=} \PY{n}{results1}\PY{o}{.}\PY{n}{V\PYZus{}out}
         \PY{p}{)}
\end{Verbatim}


\begin{Verbatim}[commandchars=\\\{\}]
{\color{outcolor}Out[{\color{outcolor}10}]:} R                                                 100000
         C                                                  1e-06
         vin    0.00    0.000000
         0.01    4.346388
         0.02    4.89{\ldots}
         dtype: object
\end{Verbatim}
            
    \begin{Verbatim}[commandchars=\\\{\}]
{\color{incolor}In [{\color{incolor}11}]:} \PY{n}{system2} \PY{o}{=} \PY{n}{make\PYZus{}system}\PY{p}{(}\PY{n}{params2}\PY{p}{)}
\end{Verbatim}


\begin{Verbatim}[commandchars=\\\{\}]
{\color{outcolor}Out[{\color{outcolor}11}]:} R                                                    100000
         C                                                     1e-06
         A                                                         5
         f                                                         1
         init                                V\_out    0
         dtype: int64
         t\_end                                                     4
         ts        [0.0, 0.01, 0.02, 0.03, 0.04, 0.05, 0.06, 0.07{\ldots}
         omega                                               6.28319
         tau                                                     0.1
         cutoff                                              1.59155
         vin       0.00    0.000000
         0.01    4.346388
         0.02    4.89{\ldots}
         dtype: object
\end{Verbatim}
            
    We create a slope function that models a high pass filter.

    \begin{Verbatim}[commandchars=\\\{\}]
{\color{incolor}In [{\color{incolor}12}]:} \PY{k}{def} \PY{n+nf}{slope\PYZus{}func\PYZus{}high\PYZus{}pass}\PY{p}{(}\PY{n}{state}\PY{p}{,} \PY{n}{t}\PY{p}{,} \PY{n}{system}\PY{p}{)}\PY{p}{:}
             \PY{l+s+sd}{\PYZdq{}\PYZdq{}\PYZdq{}Makes a slope function to update the state.}
         \PY{l+s+sd}{    }
         \PY{l+s+sd}{    state: State(V\PYZus{}out)}
         \PY{l+s+sd}{    t: time}
         \PY{l+s+sd}{    system: System Object}
         \PY{l+s+sd}{    }
         \PY{l+s+sd}{    returns: State(V\PYZus{}out)}
         \PY{l+s+sd}{    \PYZdq{}\PYZdq{}\PYZdq{}}
             \PY{n}{vout} \PY{o}{=} \PY{n}{state}
             
             \PY{n}{unpack}\PY{p}{(}\PY{n}{system}\PY{p}{)}
             
             \PY{n}{vin1} \PY{o}{=} \PY{n}{A} \PY{o}{*} \PY{n}{np}\PY{o}{.}\PY{n}{cos}\PY{p}{(}\PY{l+m+mi}{2} \PY{o}{*} \PY{n}{pi} \PY{o}{*} \PY{n}{f} \PY{o}{*} \PY{n}{t}\PY{p}{)}
             
             \PY{c+c1}{\PYZsh{}vin = interpolate(system.vin)}
             
             \PY{n}{dvindt} \PY{o}{=} \PY{p}{(}\PY{n}{vin1} \PY{o}{\PYZhy{}} \PY{n}{vout}\PY{p}{)} \PY{o}{/} \PY{p}{(}\PY{n}{R} \PY{o}{*} \PY{n}{C}\PY{p}{)}
             
             \PY{n}{dvoutdt} \PY{o}{=} \PY{n}{dvindt} \PY{o}{\PYZhy{}} \PY{p}{(}\PY{p}{(}\PY{n}{vout}\PY{p}{)} \PY{o}{/} \PY{p}{(}\PY{n}{R} \PY{o}{*} \PY{n}{C}\PY{p}{)}\PY{p}{)}
             
             \PY{k}{return} \PY{n}{dvoutdt}
\end{Verbatim}


    We create a slope function that models a low pass filter.

    \begin{Verbatim}[commandchars=\\\{\}]
{\color{incolor}In [{\color{incolor}13}]:} \PY{k}{def} \PY{n+nf}{slope\PYZus{}func\PYZus{}low\PYZus{}pass}\PY{p}{(}\PY{n}{state}\PY{p}{,} \PY{n}{t}\PY{p}{,} \PY{n}{system}\PY{p}{)}\PY{p}{:}
             \PY{l+s+sd}{\PYZdq{}\PYZdq{}\PYZdq{}Makes a slope function to update the state.}
         \PY{l+s+sd}{    }
         \PY{l+s+sd}{    state: State(V\PYZus{}out)}
         \PY{l+s+sd}{    t: time}
         \PY{l+s+sd}{    system: System Object}
         \PY{l+s+sd}{    }
         \PY{l+s+sd}{    returns: State(V\PYZus{}out)}
         \PY{l+s+sd}{    \PYZdq{}\PYZdq{}\PYZdq{}}
             \PY{n}{vout} \PY{o}{=} \PY{n}{state}
             
             \PY{n}{vin} \PY{o}{=} \PY{n}{interpolate}\PY{p}{(}\PY{n}{system}\PY{o}{.}\PY{n}{vin}\PY{p}{)}
             
             \PY{n}{dvoutdt} \PY{o}{=} \PY{p}{(}\PY{n}{vin}\PY{p}{(}\PY{n}{t}\PY{p}{)} \PY{o}{\PYZhy{}} \PY{n}{vout}\PY{p}{)} \PY{o}{/} \PY{p}{(}\PY{n}{system}\PY{o}{.}\PY{n}{R} \PY{o}{*} \PY{n}{system}\PY{o}{.}\PY{n}{C}\PY{p}{)}
             
             \PY{k}{return} \PY{n}{dvoutdt}
\end{Verbatim}


    We test both the high pass and low pass filter slope functions using the
same system object. We expect the results to be different with the same
input frequency.

    \begin{Verbatim}[commandchars=\\\{\}]
{\color{incolor}In [{\color{incolor}14}]:} \PY{n}{test\PYZus{}system} \PY{o}{=} \PY{n}{make\PYZus{}system}\PY{p}{(}\PY{n}{params1}\PY{p}{)}
\end{Verbatim}


\begin{Verbatim}[commandchars=\\\{\}]
{\color{outcolor}Out[{\color{outcolor}14}]:} R                                                      4900
         C                                                     1e-06
         A                                                         5
         f                                                         1
         init                                V\_out    0
         dtype: int64
         t\_end                                                     4
         ts        [0.0, 0.01, 0.02, 0.03, 0.04, 0.05, 0.06, 0.07{\ldots}
         omega                                               6.28319
         tau                                                  0.0049
         cutoff                                              32.4806
         vin                                                       0
         dtype: object
\end{Verbatim}
            
    \begin{Verbatim}[commandchars=\\\{\}]
{\color{incolor}In [{\color{incolor}15}]:} \PY{n}{test\PYZus{}results1}\PY{p}{,} \PY{n}{test\PYZus{}details1} \PY{o}{=} \PY{n}{run\PYZus{}ode\PYZus{}solver}\PY{p}{(}\PY{n}{test\PYZus{}system}\PY{p}{,} 
                                                       \PY{n}{slope\PYZus{}func\PYZus{}high\PYZus{}pass}\PY{p}{,} 
                                                       \PY{n}{t\PYZus{}eval}\PY{o}{=}\PY{n}{test\PYZus{}system}\PY{o}{.}\PY{n}{ts}\PY{p}{)}
\end{Verbatim}


    \begin{Verbatim}[commandchars=\\\{\}]
{\color{incolor}In [{\color{incolor}16}]:} \PY{n}{test\PYZus{}results2}\PY{p}{,} \PY{n}{test\PYZus{}details2} \PY{o}{=} \PY{n}{run\PYZus{}ode\PYZus{}solver}\PY{p}{(}\PY{n}{test\PYZus{}system}\PY{p}{,} 
                                                             \PY{n}{slope\PYZus{}func\PYZus{}init}\PY{p}{,} 
                                                             \PY{n}{t\PYZus{}eval}\PY{o}{=}\PY{n}{test\PYZus{}system}\PY{o}{.}\PY{n}{ts}\PY{p}{)}
\end{Verbatim}


    \begin{Verbatim}[commandchars=\\\{\}]
{\color{incolor}In [{\color{incolor}17}]:} \PY{n}{plot\PYZus{}results}\PY{p}{(}\PY{n}{test\PYZus{}results1}\PY{p}{)}
         \PY{n}{plot\PYZus{}results}\PY{p}{(}\PY{n}{test\PYZus{}results2}\PY{p}{)}
\end{Verbatim}


    \begin{center}
    \adjustimage{max size={0.9\linewidth}{0.9\paperheight}}{output_37_0.png}
    \end{center}
    { \hspace*{\fill} \\}
    
    The second filter is a high pass filter, and thus, we are able to use
the run\_ode\_solver function with the second set of parameters and the
high pass filter slope function.

    \begin{Verbatim}[commandchars=\\\{\}]
{\color{incolor}In [{\color{incolor}18}]:} \PY{n}{results2}\PY{p}{,} \PY{n}{details2} \PY{o}{=} \PY{n}{run\PYZus{}ode\PYZus{}solver}\PY{p}{(}\PY{n}{system2}\PY{p}{,} 
                                             \PY{n}{slope\PYZus{}func\PYZus{}high\PYZus{}pass}\PY{p}{,} 
                                             \PY{n}{t\PYZus{}eval}\PY{o}{=}\PY{n}{system2}\PY{o}{.}\PY{n}{ts}\PY{p}{)}
\end{Verbatim}


    We then set the third filter's parameters and use them to make a system
for the third filter.

    \begin{Verbatim}[commandchars=\\\{\}]
{\color{incolor}In [{\color{incolor}19}]:} \PY{n}{params3} \PY{o}{=} \PY{n}{Params}\PY{p}{(}
             \PY{n}{R} \PY{o}{=} \PY{l+m+mi}{499}\PY{p}{,}   \PY{c+c1}{\PYZsh{} ohm}
             \PY{n}{C} \PY{o}{=} \PY{l+m+mf}{10e\PYZhy{}6}\PY{p}{,}  \PY{c+c1}{\PYZsh{} farad}
             \PY{n}{vin} \PY{o}{=} \PY{n}{results2}\PY{o}{.}\PY{n}{V\PYZus{}out}
         \PY{p}{)}
\end{Verbatim}


\begin{Verbatim}[commandchars=\\\{\}]
{\color{outcolor}Out[{\color{outcolor}19}]:} R                                                    499
         C                                                  1e-05
         vin    0.00    0.000000
         0.01    0.452860
         0.02    0.82{\ldots}
         dtype: object
\end{Verbatim}
            
    \begin{Verbatim}[commandchars=\\\{\}]
{\color{incolor}In [{\color{incolor}20}]:} \PY{n}{system3} \PY{o}{=} \PY{n}{make\PYZus{}system}\PY{p}{(}\PY{n}{params3}\PY{p}{)}
\end{Verbatim}


\begin{Verbatim}[commandchars=\\\{\}]
{\color{outcolor}Out[{\color{outcolor}20}]:} R                                                       499
         C                                                     1e-05
         A                                                         5
         f                                                         1
         init                                V\_out    0
         dtype: int64
         t\_end                                                     4
         ts        [0.0, 0.01, 0.02, 0.03, 0.04, 0.05, 0.06, 0.07{\ldots}
         omega                                               6.28319
         tau                                                 0.00499
         cutoff                                              31.8948
         vin       0.00    0.000000
         0.01    0.452860
         0.02    0.82{\ldots}
         dtype: object
\end{Verbatim}
            
    The third and fourth filters used are low pass filters. Thus, we use the
same process with the low pass filter slope function for these filters.

    \begin{Verbatim}[commandchars=\\\{\}]
{\color{incolor}In [{\color{incolor}21}]:} \PY{n}{results3}\PY{p}{,} \PY{n}{details3} \PY{o}{=} \PY{n}{run\PYZus{}ode\PYZus{}solver}\PY{p}{(}\PY{n}{system3}\PY{p}{,} 
                                             \PY{n}{slope\PYZus{}func\PYZus{}low\PYZus{}pass}\PY{p}{,} 
                                             \PY{n}{t\PYZus{}eval}\PY{o}{=}\PY{n}{system3}\PY{o}{.}\PY{n}{ts}\PY{p}{)}
\end{Verbatim}


    We set the fourth filter's parameters and then use them to make a system
for the fourth filter.

    \begin{Verbatim}[commandchars=\\\{\}]
{\color{incolor}In [{\color{incolor}22}]:} \PY{n}{params4} \PY{o}{=} \PY{n}{Params}\PY{p}{(}
             \PY{n}{R} \PY{o}{=} \PY{l+m+mf}{4.9e3}\PY{p}{,}   \PY{c+c1}{\PYZsh{} ohm}
             \PY{n}{C} \PY{o}{=} \PY{l+m+mf}{1e\PYZhy{}6}\PY{p}{,}  \PY{c+c1}{\PYZsh{} farad}
             \PY{n}{vin} \PY{o}{=} \PY{n}{results3}\PY{o}{.}\PY{n}{V\PYZus{}out}
         \PY{p}{)}
\end{Verbatim}


\begin{Verbatim}[commandchars=\\\{\}]
{\color{outcolor}Out[{\color{outcolor}22}]:} R                                                   4900
         C                                                  1e-06
         vin    0.00    0.000000
         0.01    0.257302
         0.02    0.63{\ldots}
         dtype: object
\end{Verbatim}
            
    \begin{Verbatim}[commandchars=\\\{\}]
{\color{incolor}In [{\color{incolor}23}]:} \PY{n}{system4} \PY{o}{=} \PY{n}{make\PYZus{}system}\PY{p}{(}\PY{n}{params4}\PY{p}{)}
\end{Verbatim}


\begin{Verbatim}[commandchars=\\\{\}]
{\color{outcolor}Out[{\color{outcolor}23}]:} R                                                      4900
         C                                                     1e-06
         A                                                         5
         f                                                         1
         init                                V\_out    0
         dtype: int64
         t\_end                                                     4
         ts        [0.0, 0.01, 0.02, 0.03, 0.04, 0.05, 0.06, 0.07{\ldots}
         omega                                               6.28319
         tau                                                  0.0049
         cutoff                                              32.4806
         vin       0.00    0.000000
         0.01    0.257302
         0.02    0.63{\ldots}
         dtype: object
\end{Verbatim}
            
    \begin{Verbatim}[commandchars=\\\{\}]
{\color{incolor}In [{\color{incolor}24}]:} \PY{n}{results4}\PY{p}{,} \PY{n}{details4} \PY{o}{=} \PY{n}{run\PYZus{}ode\PYZus{}solver}\PY{p}{(}\PY{n}{system4}\PY{p}{,} 
                                             \PY{n}{slope\PYZus{}func\PYZus{}low\PYZus{}pass}\PY{p}{,} 
                                             \PY{n}{t\PYZus{}eval}\PY{o}{=}\PY{n}{system4}\PY{o}{.}\PY{n}{ts}\PY{p}{)}
\end{Verbatim}


    \subsubsection{Results 1:}\label{results-1}

    We plot the results of all of the filters on the same chart.

    \begin{Verbatim}[commandchars=\\\{\}]
{\color{incolor}In [{\color{incolor}25}]:} \PY{n}{plot\PYZus{}results}\PY{p}{(}\PY{n}{results1}\PY{p}{)}
         \PY{n}{plot\PYZus{}results}\PY{p}{(}\PY{n}{results2}\PY{p}{)}
         \PY{n}{plot\PYZus{}results}\PY{p}{(}\PY{n}{results3}\PY{p}{)}
         \PY{n}{plot\PYZus{}results}\PY{p}{(}\PY{n}{results4}\PY{p}{)}
\end{Verbatim}


    \begin{center}
    \adjustimage{max size={0.9\linewidth}{0.9\paperheight}}{output_51_0.png}
    \end{center}
    { \hspace*{\fill} \\}
    
    We then plot the input signal and the output signal to compare the
attenuation through the circuit.

    \begin{Verbatim}[commandchars=\\\{\}]
{\color{incolor}In [{\color{incolor}26}]:} \PY{c+c1}{\PYZsh{}plt.xkcd()}
         \PY{n}{plot\PYZus{}results}\PY{p}{(}\PY{n}{results4}\PY{p}{)}
         \PY{n}{plt}\PY{o}{.}\PY{n}{plot}\PY{p}{(}\PY{n}{x}\PY{p}{,} \PY{n}{v}\PY{p}{)}
         \PY{n}{plt}\PY{o}{.}\PY{n}{legend}\PY{p}{(}\PY{p}{[}\PY{l+s+s2}{\PYZdq{}}\PY{l+s+s2}{Filtered Signal}\PY{l+s+s2}{\PYZdq{}}\PY{p}{,} \PY{l+s+s2}{\PYZdq{}}\PY{l+s+s2}{Input Signal}\PY{l+s+s2}{\PYZdq{}}\PY{p}{]}\PY{p}{,} 
                    \PY{n}{loc}\PY{o}{=}\PY{l+s+s1}{\PYZsq{}}\PY{l+s+s1}{upper right}\PY{l+s+s1}{\PYZsq{}}\PY{p}{,} \PY{n}{bbox\PYZus{}to\PYZus{}anchor}\PY{o}{=}\PY{p}{(}\PY{l+m+mf}{1.35}\PY{p}{,} \PY{l+m+mi}{1}\PY{p}{)}\PY{p}{)}
\end{Verbatim}


\begin{Verbatim}[commandchars=\\\{\}]
{\color{outcolor}Out[{\color{outcolor}26}]:} <matplotlib.legend.Legend at 0x1bd1165d7f0>
\end{Verbatim}
            
    \begin{center}
    \adjustimage{max size={0.9\linewidth}{0.9\paperheight}}{output_53_1.png}
    \end{center}
    { \hspace*{\fill} \\}
    
    \subsubsection{Model 2:}\label{model-2}

    We define a function that determines the amplification of the AD623
integrated circuit used. This value is expressed through the equation,

\[G = (1+ \frac{100 k\Omega}{R_g})\]

Where \(R_g\) is the resistor across pins 1 and 8.

    \begin{Verbatim}[commandchars=\\\{\}]
{\color{incolor}In [{\color{incolor}27}]:} \PY{k}{def} \PY{n+nf}{amp}\PY{p}{(}\PY{n}{R}\PY{p}{)}\PY{p}{:}
             \PY{l+s+sd}{\PYZdq{}\PYZdq{}\PYZdq{}Makes function to determine the gain of the AD623 amplifier.}
         \PY{l+s+sd}{    }
         \PY{l+s+sd}{    R: The reststance value across pins 1 and 8 on this chip}
         \PY{l+s+sd}{    }
         \PY{l+s+sd}{    returns: The gain of the amplifier}
         \PY{l+s+sd}{    \PYZdq{}\PYZdq{}\PYZdq{}}
             \PY{n}{G} \PY{o}{=} \PY{l+m+mi}{1} \PY{o}{+} \PY{p}{(}\PY{l+m+mf}{100e3}\PY{o}{/}\PY{n}{R}\PY{p}{)}
             \PY{k}{return} \PY{n}{G}
\end{Verbatim}


    We find the gain for both amplifiers. G1 expresses the gain of the first
chip, and G2 expresses the gain of the second chip.

    \begin{Verbatim}[commandchars=\\\{\}]
{\color{incolor}In [{\color{incolor}28}]:} \PY{n}{G1} \PY{o}{=} \PY{n}{amp}\PY{p}{(}\PY{l+m+mf}{2e3}\PY{p}{)}
         \PY{n}{G2} \PY{o}{=} \PY{n}{amp}\PY{p}{(}\PY{l+m+mf}{4.9e3}\PY{p}{)}\PY{p}{;}
\end{Verbatim}


    We set the parameters of all the resisters and capacitors in the
circuit, which we then use in a function that runs the entire circuit
simulation.

    \begin{Verbatim}[commandchars=\\\{\}]
{\color{incolor}In [{\color{incolor}29}]:} \PY{n}{run\PYZus{}sim\PYZus{}system} \PY{o}{=} \PY{n}{System}\PY{p}{(}\PY{n}{R1} \PY{o}{=} \PY{l+m+mf}{4.9e3}\PY{p}{,} \PY{c+c1}{\PYZsh{} ohm}
                                 \PY{n}{C1} \PY{o}{=} \PY{l+m+mf}{1e\PYZhy{}6}\PY{p}{,}  \PY{c+c1}{\PYZsh{} farad }
                                 \PY{n}{A} \PY{o}{=} \PY{l+m+mi}{5} \PY{o}{*} \PY{n}{G1}\PY{p}{,} \PY{c+c1}{\PYZsh{} volt}
                                 \PY{n}{f} \PY{o}{=} \PY{n}{input\PYZus{}freq}\PY{o}{.}\PY{n}{f}\PY{p}{,}   \PY{c+c1}{\PYZsh{} Hz }
                                 \PY{n}{vin} \PY{o}{=} \PY{l+m+mi}{0}\PY{p}{,} 
                                 \PY{n}{R2} \PY{o}{=} \PY{l+m+mf}{100e3}\PY{p}{,}   \PY{c+c1}{\PYZsh{} ohm}
                                 \PY{n}{C2} \PY{o}{=} \PY{l+m+mf}{1e\PYZhy{}6}\PY{p}{,}  \PY{c+c1}{\PYZsh{} farad}
                                 \PY{n}{R3} \PY{o}{=} \PY{l+m+mi}{499}\PY{p}{,}   \PY{c+c1}{\PYZsh{} ohm}
                                 \PY{n}{C3} \PY{o}{=} \PY{l+m+mf}{10e\PYZhy{}6}\PY{p}{,}  \PY{c+c1}{\PYZsh{} farad}
                                 \PY{n}{R4} \PY{o}{=} \PY{l+m+mf}{4.9e3}\PY{p}{,}   \PY{c+c1}{\PYZsh{} ohm}
                                 \PY{n}{C4} \PY{o}{=} \PY{l+m+mf}{1e\PYZhy{}6}  \PY{c+c1}{\PYZsh{} farad}
                                 \PY{p}{)}
\end{Verbatim}


\begin{Verbatim}[commandchars=\\\{\}]
{\color{outcolor}Out[{\color{outcolor}29}]:} R1       4900.000000
         C1          0.000001
         A         255.000000
         f           1.000000
         vin         0.000000
         R2     100000.000000
         C2          0.000001
         R3        499.000000
         C3          0.000010
         R4       4900.000000
         C4          0.000001
         dtype: float64
\end{Verbatim}
            
    We define a function that runs through the entire simulation using the
functions defined in Model 1 and the parameters defined above.

    \begin{Verbatim}[commandchars=\\\{\}]
{\color{incolor}In [{\color{incolor}30}]:} \PY{k}{def} \PY{n+nf}{run\PYZus{}sim}\PY{p}{(}\PY{n}{input\PYZus{}freq}\PY{p}{,} \PY{n}{make\PYZus{}system}\PY{p}{,} 
                     \PY{n}{slope\PYZus{}func\PYZus{}init}\PY{p}{,} \PY{n}{slope\PYZus{}func\PYZus{}high\PYZus{}pass}\PY{p}{,} 
                     \PY{n}{slope\PYZus{}func\PYZus{}low\PYZus{}pass}\PY{p}{,} \PY{n}{G1}\PY{p}{,} \PY{n}{G2}\PY{p}{,} \PY{n}{system}\PY{p}{)}\PY{p}{:}
             
             \PY{l+s+sd}{\PYZdq{}\PYZdq{}\PYZdq{}Define a function that runs the enitre circuit simulation}
         \PY{l+s+sd}{    }
         \PY{l+s+sd}{    inputs}
         \PY{l+s+sd}{        input\PYZus{}freq: the input signal frequency}
         \PY{l+s+sd}{        make\PYZus{}system: function that takes parameters and returns a system}
         \PY{l+s+sd}{        slope\PYZus{}func\PYZus{}init: used to model the first filter}
         \PY{l+s+sd}{        slope\PYZus{}func\PYZus{}high\PYZus{}pass: used to model high pass filters}
         \PY{l+s+sd}{        slope\PYZus{}func\PYZus{}low\PYZus{}pass: used to model low pass filters}
         \PY{l+s+sd}{        G1: the gain of the first amplifier}
         \PY{l+s+sd}{        G2: the gain of the second amplifier}
         \PY{l+s+sd}{        system: the parameters of the circuit}
         \PY{l+s+sd}{    }
         \PY{l+s+sd}{    returns}
         \PY{l+s+sd}{        results5: the output of the first filter}
         \PY{l+s+sd}{        results6: the output of the second filter}
         \PY{l+s+sd}{        results7: the output of the thrid filter}
         \PY{l+s+sd}{        results8: the output of the fourth filter}
         \PY{l+s+sd}{        ratio: the ratio of input signal amplitude to ourput signal amplitude}
         \PY{l+s+sd}{    \PYZdq{}\PYZdq{}\PYZdq{}}
             
             \PY{n}{params5} \PY{o}{=} \PY{n}{Params}\PY{p}{(}
             \PY{n}{R} \PY{o}{=} \PY{n}{system}\PY{o}{.}\PY{n}{R1}\PY{p}{,}   
             \PY{n}{C} \PY{o}{=} \PY{n}{system}\PY{o}{.}\PY{n}{C1}\PY{p}{,}  
             \PY{n}{A} \PY{o}{=} \PY{n}{system}\PY{o}{.}\PY{n}{A}\PY{p}{,}      
             \PY{n}{f} \PY{o}{=} \PY{n}{system}\PY{o}{.}\PY{n}{f}\PY{p}{,}   
             \PY{n}{vin} \PY{o}{=} \PY{n}{system}\PY{o}{.}\PY{n}{vin}
             \PY{p}{)}
             
             \PY{n}{system5} \PY{o}{=} \PY{n}{make\PYZus{}system}\PY{p}{(}\PY{n}{params5}\PY{p}{)}
             \PY{n}{results5}\PY{p}{,} \PY{n}{details5} \PY{o}{=} \PY{n}{run\PYZus{}ode\PYZus{}solver}\PY{p}{(}\PY{n}{system5}\PY{p}{,} 
                                                 \PY{n}{slope\PYZus{}func\PYZus{}init}\PY{p}{,} 
                                                 \PY{n}{t\PYZus{}eval}\PY{o}{=}\PY{n}{system5}\PY{o}{.}\PY{n}{ts}\PY{p}{)}
             
             \PY{n}{params6} \PY{o}{=} \PY{n}{Params}\PY{p}{(}
             \PY{n}{R} \PY{o}{=} \PY{n}{system}\PY{o}{.}\PY{n}{R2}\PY{p}{,}
             \PY{n}{C} \PY{o}{=} \PY{n}{system}\PY{o}{.}\PY{n}{C2}\PY{p}{,}
             \PY{n}{vin} \PY{o}{=} \PY{n}{results5}\PY{o}{.}\PY{n}{V\PYZus{}out}
             \PY{p}{)}
             
             \PY{n}{system6} \PY{o}{=} \PY{n}{make\PYZus{}system}\PY{p}{(}\PY{n}{params6}\PY{p}{)}
             \PY{n}{results6}\PY{p}{,} \PY{n}{details6} \PY{o}{=} \PY{n}{run\PYZus{}ode\PYZus{}solver}\PY{p}{(}\PY{n}{system6}\PY{p}{,} 
                                                 \PY{n}{slope\PYZus{}func\PYZus{}high\PYZus{}pass}\PY{p}{,} 
                                                 \PY{n}{t\PYZus{}eval}\PY{o}{=}\PY{n}{system6}\PY{o}{.}\PY{n}{ts}\PY{p}{)}
             
             \PY{n}{params7} \PY{o}{=} \PY{n}{Params}\PY{p}{(}
             \PY{n}{R} \PY{o}{=} \PY{n}{system}\PY{o}{.}\PY{n}{R3}\PY{p}{,}
             \PY{n}{C} \PY{o}{=} \PY{n}{system}\PY{o}{.}\PY{n}{C3}\PY{p}{,}
             \PY{n}{vin} \PY{o}{=} \PY{n}{results6}\PY{o}{.}\PY{n}{V\PYZus{}out} \PY{o}{*} \PY{n}{G2}
             \PY{p}{)}
             
             \PY{n}{system7} \PY{o}{=} \PY{n}{make\PYZus{}system}\PY{p}{(}\PY{n}{params7}\PY{p}{)}
             \PY{n}{results7}\PY{p}{,} \PY{n}{details7} \PY{o}{=} \PY{n}{run\PYZus{}ode\PYZus{}solver}\PY{p}{(}\PY{n}{system7}\PY{p}{,} 
                                                 \PY{n}{slope\PYZus{}func\PYZus{}low\PYZus{}pass}\PY{p}{,} 
                                                 \PY{n}{t\PYZus{}eval}\PY{o}{=}\PY{n}{system7}\PY{o}{.}\PY{n}{ts}\PY{p}{)}
             
             \PY{n}{params8} \PY{o}{=} \PY{n}{Params}\PY{p}{(}
             \PY{n}{R} \PY{o}{=} \PY{n}{system}\PY{o}{.}\PY{n}{R4}\PY{p}{,}
             \PY{n}{C} \PY{o}{=} \PY{n}{system}\PY{o}{.}\PY{n}{C4}\PY{p}{,}
             \PY{n}{vin} \PY{o}{=} \PY{n}{results7}\PY{o}{.}\PY{n}{V\PYZus{}out}
             \PY{p}{)}
             
             \PY{n}{system8} \PY{o}{=} \PY{n}{make\PYZus{}system}\PY{p}{(}\PY{n}{params8}\PY{p}{)}
             \PY{n}{results8}\PY{p}{,} \PY{n}{details8} \PY{o}{=} \PY{n}{run\PYZus{}ode\PYZus{}solver}\PY{p}{(}\PY{n}{system8}\PY{p}{,} 
                                                 \PY{n}{slope\PYZus{}func\PYZus{}low\PYZus{}pass}\PY{p}{,} 
                                                 \PY{n}{t\PYZus{}eval}\PY{o}{=}\PY{n}{system8}\PY{o}{.}\PY{n}{ts}\PY{p}{)}
             
             \PY{n}{A\PYZus{}in} \PY{o}{=} \PY{n}{params5}\PY{o}{.}\PY{n}{A}
             
             \PY{n}{A\PYZus{}out}\PY{o}{=}\PY{p}{(}\PY{n}{results8}\PY{o}{.}\PY{n}{V\PYZus{}out}\PY{o}{.}\PY{n}{max}\PY{p}{(}\PY{p}{)}\PY{o}{\PYZhy{}}\PY{n}{results8}\PY{o}{.}\PY{n}{V\PYZus{}out}\PY{o}{.}\PY{n}{min}\PY{p}{(}\PY{p}{)}\PY{p}{)}\PY{o}{/}\PY{l+m+mi}{2}
                 
             \PY{n}{ratio} \PY{o}{=} \PY{n}{A\PYZus{}out}\PY{o}{/}\PY{n}{A\PYZus{}in}
             
             \PY{k}{return} \PY{n}{State}\PY{p}{(}\PY{n}{results5}\PY{o}{=}\PY{n}{results5}\PY{p}{,} 
                          \PY{n}{results6}\PY{o}{=}\PY{n}{results6}\PY{p}{,} 
                          \PY{n}{result7}\PY{o}{=}\PY{n}{results7}\PY{p}{,} 
                          \PY{n}{results8}\PY{o}{=}\PY{n}{results8}\PY{p}{,} 
                          \PY{n}{ratio}\PY{o}{=}\PY{n}{ratio}\PY{p}{)}
\end{Verbatim}


    We set the input frequency to 50 Hz, to test the circuit at high
frequencies. We do this to confirm that each of the filters are working
as there is an apparent difference in the outputs of each filter.

    \begin{Verbatim}[commandchars=\\\{\}]
{\color{incolor}In [{\color{incolor}31}]:} \PY{n}{input\PYZus{}freq}\PY{o}{.}\PY{n}{f} \PY{o}{=} \PY{l+m+mi}{50}
\end{Verbatim}


    \begin{Verbatim}[commandchars=\\\{\}]
{\color{incolor}In [{\color{incolor}32}]:} \PY{n}{results5}\PY{p}{,} \PY{n}{results6}\PY{p}{,} \PY{n}{results7}\PY{p}{,} \PY{n}{results8}\PY{p}{,} \PY{n}{ratio} \PY{o}{=} \PY{n}{run\PYZus{}sim}\PY{p}{(}\PY{n}{input\PYZus{}freq}\PY{p}{,} 
                                                                 \PY{n}{make\PYZus{}system}\PY{p}{,} 
                                                                 \PY{n}{slope\PYZus{}func\PYZus{}init}\PY{p}{,} 
                                                                 \PY{n}{slope\PYZus{}func\PYZus{}high\PYZus{}pass}\PY{p}{,} 
                                                                 \PY{n}{slope\PYZus{}func\PYZus{}low\PYZus{}pass}\PY{p}{,} 
                                                                 \PY{n}{G1}\PY{p}{,} \PY{n}{G2}\PY{p}{,} \PY{n}{run\PYZus{}sim\PYZus{}system}\PY{p}{)}
\end{Verbatim}


    \subsubsection{Results 2:}\label{results-2}

    The simulation is run with an input frequency of 50 Hz to show that each
of the filters are working. We can see that much of the high frequency
is filtered out.

    \begin{Verbatim}[commandchars=\\\{\}]
{\color{incolor}In [{\color{incolor}33}]:} \PY{n}{plot\PYZus{}results}\PY{p}{(}\PY{n}{results5}\PY{p}{)}
         \PY{n}{plot\PYZus{}results}\PY{p}{(}\PY{n}{results6}\PY{p}{)}
         \PY{n}{plot\PYZus{}results}\PY{p}{(}\PY{n}{results7}\PY{p}{)}
         \PY{n}{plot\PYZus{}results}\PY{p}{(}\PY{n}{results8}\PY{p}{)}
\end{Verbatim}


    \begin{center}
    \adjustimage{max size={0.9\linewidth}{0.9\paperheight}}{output_68_0.png}
    \end{center}
    { \hspace*{\fill} \\}
    
    \subsubsection{Interpretation:}\label{interpretation}

    We sweep the input signal frequencies to create a Bode Plot of the
system. We are able to compare the results of our simulated model to the
experimental data.

    \begin{Verbatim}[commandchars=\\\{\}]
{\color{incolor}In [{\color{incolor}34}]:} \PY{n}{sweep} \PY{o}{=} \PY{n}{SweepSeries}\PY{p}{(}\PY{p}{)}
         \PY{k}{for} \PY{n}{f} \PY{o+ow}{in} \PY{n}{linspace}\PY{p}{(}\PY{l+m+mi}{1}\PY{p}{,} \PY{l+m+mi}{100}\PY{p}{,} \PY{l+m+mi}{50}\PY{p}{)}\PY{p}{:}
             \PY{n}{input\PYZus{}freq}\PY{o}{.}\PY{n}{f}\PY{o}{=}\PY{n}{f}
             \PY{n}{state} \PY{o}{=} \PY{n}{run\PYZus{}sim}\PY{p}{(}\PY{n}{input\PYZus{}freq}\PY{p}{,} \PY{n}{make\PYZus{}system}\PY{p}{,} 
                             \PY{n}{slope\PYZus{}func\PYZus{}init}\PY{p}{,} 
                             \PY{n}{slope\PYZus{}func\PYZus{}high\PYZus{}pass}\PY{p}{,} 
                             \PY{n}{slope\PYZus{}func\PYZus{}low\PYZus{}pass}\PY{p}{,} 
                             \PY{n}{G1}\PY{p}{,} \PY{n}{G2}\PY{p}{,} \PY{n}{run\PYZus{}sim\PYZus{}system}\PY{p}{)}\PY{o}{.}\PY{n}{ratio}
             \PY{n}{sweep}\PY{p}{[}\PY{n}{f}\PY{p}{]} \PY{o}{=} \PY{n}{state}\PY{p}{;}
\end{Verbatim}


    We plot the results of our simulated Bode Plot with frequencies swept
from 1 Hz to 100 Hz.

    \begin{Verbatim}[commandchars=\\\{\}]
{\color{incolor}In [{\color{incolor}35}]:} \PY{n}{plot}\PY{p}{(}\PY{n}{sweep}\PY{p}{)}
         \PY{n}{decorate}\PY{p}{(}\PY{n}{xlabel}\PY{o}{=}\PY{l+s+s1}{\PYZsq{}}\PY{l+s+s1}{Frequency (Hz)}\PY{l+s+s1}{\PYZsq{}}\PY{p}{,}
                      \PY{n}{ylabel}\PY{o}{=}\PY{l+s+s1}{\PYZsq{}}\PY{l+s+s1}{Amplitude}\PY{l+s+s1}{\PYZsq{}}\PY{p}{,}
                      \PY{n}{xscale}\PY{o}{=}\PY{l+s+s1}{\PYZsq{}}\PY{l+s+s1}{log}\PY{l+s+s1}{\PYZsq{}}\PY{p}{,}
                      \PY{n}{yscale}\PY{o}{=}\PY{l+s+s1}{\PYZsq{}}\PY{l+s+s1}{log}\PY{l+s+s1}{\PYZsq{}}\PY{p}{)}
\end{Verbatim}


    \begin{center}
    \adjustimage{max size={0.9\linewidth}{0.9\paperheight}}{output_73_0.png}
    \end{center}
    { \hspace*{\fill} \\}
    
    We import the experimental data.

    \begin{Verbatim}[commandchars=\\\{\}]
{\color{incolor}In [{\color{incolor}36}]:} \PY{k+kn}{from} \PY{n+nn}{pandas} \PY{k}{import} \PY{n}{read\PYZus{}csv}
         \PY{n}{data} \PY{o}{=} \PY{n}{read\PYZus{}csv}\PY{p}{(}\PY{l+s+s2}{\PYZdq{}}\PY{l+s+s2}{Bode Plot Data Two Channel Adjusted.csv}\PY{l+s+s2}{\PYZdq{}}\PY{p}{)}
         \PY{n}{data}\PY{o}{.}\PY{n}{columns} \PY{o}{=} \PY{p}{[}\PY{l+s+s2}{\PYZdq{}}\PY{l+s+s2}{Frequency}\PY{l+s+s2}{\PYZdq{}}\PY{p}{,} \PY{l+s+s2}{\PYZdq{}}\PY{l+s+s2}{Channel1}\PY{l+s+s2}{\PYZdq{}}\PY{p}{,} \PY{l+s+s2}{\PYZdq{}}\PY{l+s+s2}{Channel2}\PY{l+s+s2}{\PYZdq{}}\PY{p}{]}
         \PY{n+nb}{print}\PY{p}{(}\PY{n}{data}\PY{p}{)}
\end{Verbatim}


    \begin{Verbatim}[commandchars=\\\{\}]
    Frequency  Channel1   Channel2
0    1.204622  1.336704   7.892721
1    1.394753  2.062281   7.447578
2    1.614894  2.720442   6.957306
3    1.869782  3.309978   6.506548
4    2.164899  3.832494   6.041254
5    2.506597  4.289950   5.543557
6    2.902226  4.685360   5.103302
7    3.360299  5.021772   4.687570
8    3.890673  5.305573   4.287470
9    4.504759  5.534682   3.906275
10   5.215768  5.710039   3.533519
11   6.039000  5.834386   3.155440
12   6.992167  5.909790   2.756838
13   8.095778  5.936137   2.324386
14   9.373577  5.912632   1.843787
15  10.853058  5.834932   1.306494
16  12.566053  5.702385   0.701586
17  14.549419  5.513495   0.023472
18  16.845831  5.264004  -0.726697
19  19.504696  4.946740  -1.541431
20  22.583225  4.554394  -2.408142
21  26.147654  4.080540  -3.317242
22  30.274676  3.520765  -4.257796
23  35.053087  2.875092  -5.225047
24  40.585699  2.145339  -6.203025
25  46.991553  1.336644  -7.194759
26  54.408477  0.455705  -8.204307
27  62.996052 -0.488923  -9.210187
28  72.939050 -1.488359 -10.235730
29  84.451403 -2.535038 -11.283204
30  97.780812 -3.619814 -12.382250

    \end{Verbatim}

    \begin{Verbatim}[commandchars=\\\{\}]
{\color{incolor}In [{\color{incolor}37}]:} \PY{k}{def} \PY{n+nf}{plot\PYZus{}data}\PY{p}{(}\PY{n}{data}\PY{p}{)}\PY{p}{:}
             \PY{l+s+sd}{\PYZdq{}\PYZdq{}\PYZdq{}We define a function that plots the Bode Plot data}
         \PY{l+s+sd}{    }
         \PY{l+s+sd}{    data: the experimental results}
         \PY{l+s+sd}{    }
         \PY{l+s+sd}{    returns: a Bode Plot of experimental data}
         \PY{l+s+sd}{    \PYZdq{}\PYZdq{}\PYZdq{}}
             \PY{n}{f} \PY{o}{=} \PY{p}{(}\PY{n}{data}\PY{o}{.}\PY{n}{Frequency}\PY{p}{)}
             \PY{n}{a} \PY{o}{=} \PY{p}{(}\PY{n}{data}\PY{o}{.}\PY{n}{Channel2}\PY{p}{)}
                
             \PY{n}{plot}\PY{p}{(}\PY{n}{f}\PY{p}{,} \PY{n}{a}\PY{p}{)}
             \PY{n}{decorate}\PY{p}{(}\PY{n}{xlabel}\PY{o}{=}\PY{l+s+s1}{\PYZsq{}}\PY{l+s+s1}{Frequency (Hz)}\PY{l+s+s1}{\PYZsq{}}\PY{p}{,}
                      \PY{n}{ylabel}\PY{o}{=}\PY{l+s+s1}{\PYZsq{}}\PY{l+s+s1}{Amplitude}\PY{l+s+s1}{\PYZsq{}}\PY{p}{,}
                      \PY{n}{xscale}\PY{o}{=}\PY{l+s+s1}{\PYZsq{}}\PY{l+s+s1}{log}\PY{l+s+s1}{\PYZsq{}}\PY{p}{,}
                      \PY{n}{yscale}\PY{o}{=}\PY{l+s+s1}{\PYZsq{}}\PY{l+s+s1}{log}\PY{l+s+s1}{\PYZsq{}}\PY{p}{,}
                      \PY{n}{title}\PY{o}{=}\PY{l+s+s1}{\PYZsq{}}\PY{l+s+s1}{Bode Plot Collected Data}\PY{l+s+s1}{\PYZsq{}}\PY{p}{,}
                      \PY{n}{legend}\PY{o}{=}\PY{k+kc}{False}\PY{p}{)}
\end{Verbatim}


    We plot the experimental results.

    \begin{Verbatim}[commandchars=\\\{\}]
{\color{incolor}In [{\color{incolor}38}]:} \PY{n}{plot\PYZus{}data}\PY{p}{(}\PY{n}{data}\PY{p}{)}
\end{Verbatim}


    \begin{center}
    \adjustimage{max size={0.9\linewidth}{0.9\paperheight}}{output_78_0.png}
    \end{center}
    { \hspace*{\fill} \\}
    
    The experimental and theoretical results are graphed on the same plot.
We can see that the experimental results show more harsh attenuation
near the cutoff frequency; whereas, the theoretical results have a much
more shallow cutoff.

    \begin{Verbatim}[commandchars=\\\{\}]
{\color{incolor}In [{\color{incolor}39}]:} \PY{n}{plot}\PY{p}{(}\PY{n}{sweep}\PY{p}{)}
         \PY{n}{decorate}\PY{p}{(}\PY{n}{xlabel}\PY{o}{=}\PY{l+s+s1}{\PYZsq{}}\PY{l+s+s1}{Frequency (Hz)}\PY{l+s+s1}{\PYZsq{}}\PY{p}{,}
                      \PY{n}{ylabel}\PY{o}{=}\PY{l+s+s1}{\PYZsq{}}\PY{l+s+s1}{Amplitude}\PY{l+s+s1}{\PYZsq{}}\PY{p}{,}
                      \PY{n}{title}\PY{o}{=}\PY{l+s+s1}{\PYZsq{}}\PY{l+s+s1}{Bode Plot Simulated VS Collected Data}\PY{l+s+s1}{\PYZsq{}}\PY{p}{)}
         \PY{n}{plot\PYZus{}data}\PY{p}{(}\PY{n}{data}\PY{p}{)}
\end{Verbatim}


    \begin{center}
    \adjustimage{max size={0.9\linewidth}{0.9\paperheight}}{output_80_0.png}
    \end{center}
    { \hspace*{\fill} \\}
    
    \subsubsection{Model 3:}\label{model-3}

    We create parameters to determine how the model will respond to various
changes in R and C values.

    \begin{Verbatim}[commandchars=\\\{\}]
{\color{incolor}In [{\color{incolor}40}]:} \PY{n}{input\PYZus{}freq}\PY{o}{.}\PY{n}{f} \PY{o}{=} \PY{l+m+mi}{1}
\end{Verbatim}


    These are the first set of R and C values we tested.

    \begin{Verbatim}[commandchars=\\\{\}]
{\color{incolor}In [{\color{incolor}41}]:} \PY{n}{run\PYZus{}sim\PYZus{}system2} \PY{o}{=} \PY{n}{System}\PY{p}{(}\PY{n}{R1} \PY{o}{=} \PY{l+m+mf}{3.01e4}\PY{p}{,} \PY{c+c1}{\PYZsh{} ohm}
                                  \PY{n}{C1} \PY{o}{=} \PY{l+m+mf}{1e\PYZhy{}6}\PY{p}{,}  \PY{c+c1}{\PYZsh{} farad }
                                  \PY{n}{A} \PY{o}{=} \PY{l+m+mi}{5} \PY{o}{*} \PY{n}{G1}\PY{p}{,} \PY{c+c1}{\PYZsh{} volt}
                                  \PY{n}{f} \PY{o}{=} \PY{n}{input\PYZus{}freq}\PY{o}{.}\PY{n}{f}\PY{p}{,}   \PY{c+c1}{\PYZsh{} Hz }
                                  \PY{n}{vin} \PY{o}{=} \PY{l+m+mi}{0}\PY{p}{,} 
                                  \PY{n}{R2} \PY{o}{=} \PY{l+m+mf}{1e5}\PY{p}{,}   \PY{c+c1}{\PYZsh{} ohm}
                                  \PY{n}{C2} \PY{o}{=} \PY{l+m+mf}{3.3e\PYZhy{}6}\PY{p}{,}  \PY{c+c1}{\PYZsh{} farad}
                                  \PY{n}{R3} \PY{o}{=} \PY{l+m+mf}{1e6}\PY{p}{,}   \PY{c+c1}{\PYZsh{} ohm}
                                  \PY{n}{C3} \PY{o}{=} \PY{l+m+mf}{1e\PYZhy{}7}\PY{p}{,}  \PY{c+c1}{\PYZsh{} farad}
                                  \PY{n}{R4} \PY{o}{=} \PY{l+m+mf}{4.99e3}\PY{p}{,}   \PY{c+c1}{\PYZsh{} ohm}
                                  \PY{n}{C4} \PY{o}{=} \PY{l+m+mf}{1e\PYZhy{}5}  \PY{c+c1}{\PYZsh{} farad}
                                 \PY{p}{)}
\end{Verbatim}


\begin{Verbatim}[commandchars=\\\{\}]
{\color{outcolor}Out[{\color{outcolor}41}]:} R1     3.010000e+04
         C1     1.000000e-06
         A      2.550000e+02
         f      1.000000e+00
         vin    0.000000e+00
         R2     1.000000e+05
         C2     3.300000e-06
         R3     1.000000e+06
         C3     1.000000e-07
         R4     4.990000e+03
         C4     1.000000e-05
         dtype: float64
\end{Verbatim}
            
    \begin{Verbatim}[commandchars=\\\{\}]
{\color{incolor}In [{\color{incolor}42}]:} \PY{n}{results5}\PY{p}{,} \PY{n}{results6}\PY{p}{,} \PY{n}{results7}\PY{p}{,} \PY{n}{results8}\PY{p}{,} \PY{n}{ratio} \PY{o}{=} \PY{n}{run\PYZus{}sim}\PY{p}{(}\PY{n}{input\PYZus{}freq}\PY{p}{,} 
                                                                 \PY{n}{make\PYZus{}system}\PY{p}{,} 
                                                                 \PY{n}{slope\PYZus{}func\PYZus{}init}\PY{p}{,} 
                                                                 \PY{n}{slope\PYZus{}func\PYZus{}high\PYZus{}pass}\PY{p}{,} 
                                                                 \PY{n}{slope\PYZus{}func\PYZus{}low\PYZus{}pass}\PY{p}{,} 
                                                                 \PY{n}{G1}\PY{p}{,} \PY{n}{G2}\PY{p}{,} \PY{n}{run\PYZus{}sim\PYZus{}system2}\PY{p}{)}
\end{Verbatim}


    \begin{Verbatim}[commandchars=\\\{\}]
{\color{incolor}In [{\color{incolor}43}]:} \PY{n}{plot\PYZus{}results}\PY{p}{(}\PY{n}{results5}\PY{p}{)}
         \PY{n}{plot\PYZus{}results}\PY{p}{(}\PY{n}{results6}\PY{p}{)}
         \PY{n}{plot\PYZus{}results}\PY{p}{(}\PY{n}{results7}\PY{p}{)}
         \PY{n}{plot\PYZus{}results}\PY{p}{(}\PY{n}{results8}\PY{p}{)}
\end{Verbatim}


    \begin{center}
    \adjustimage{max size={0.9\linewidth}{0.9\paperheight}}{output_87_0.png}
    \end{center}
    { \hspace*{\fill} \\}
    
    \begin{Verbatim}[commandchars=\\\{\}]
{\color{incolor}In [{\color{incolor}44}]:} \PY{n}{sweep2} \PY{o}{=} \PY{n}{SweepSeries}\PY{p}{(}\PY{p}{)}
         \PY{k}{for} \PY{n}{f} \PY{o+ow}{in} \PY{n}{linspace}\PY{p}{(}\PY{l+m+mi}{1}\PY{p}{,} \PY{l+m+mi}{100}\PY{p}{,} \PY{l+m+mi}{50}\PY{p}{)}\PY{p}{:}
             \PY{n}{input\PYZus{}freq}\PY{o}{.}\PY{n}{f}\PY{o}{=}\PY{n}{f}
             \PY{n}{state} \PY{o}{=} \PY{n}{run\PYZus{}sim}\PY{p}{(}\PY{n}{input\PYZus{}freq}\PY{p}{,} \PY{n}{make\PYZus{}system}\PY{p}{,} 
                             \PY{n}{slope\PYZus{}func\PYZus{}init}\PY{p}{,} 
                             \PY{n}{slope\PYZus{}func\PYZus{}high\PYZus{}pass}\PY{p}{,} 
                             \PY{n}{slope\PYZus{}func\PYZus{}low\PYZus{}pass}\PY{p}{,} 
                             \PY{n}{G1}\PY{p}{,} \PY{n}{G2}\PY{p}{,} \PY{n}{run\PYZus{}sim\PYZus{}system2}\PY{p}{)}\PY{o}{.}\PY{n}{ratio}
             \PY{n}{sweep2}\PY{p}{[}\PY{n}{f}\PY{p}{]} \PY{o}{=} \PY{n}{state}\PY{p}{;}
\end{Verbatim}


    These are the second set of R and C values we tested.

    \begin{Verbatim}[commandchars=\\\{\}]
{\color{incolor}In [{\color{incolor}45}]:} \PY{n}{run\PYZus{}sim\PYZus{}system3} \PY{o}{=} \PY{n}{System}\PY{p}{(}\PY{n}{R1} \PY{o}{=} \PY{l+m+mi}{604}\PY{p}{,} \PY{c+c1}{\PYZsh{} ohm}
                                  \PY{n}{C1} \PY{o}{=} \PY{l+m+mf}{1e\PYZhy{}4}\PY{p}{,}  \PY{c+c1}{\PYZsh{} farad }
                                  \PY{n}{A} \PY{o}{=} \PY{l+m+mi}{5} \PY{o}{*} \PY{n}{G1}\PY{p}{,} \PY{c+c1}{\PYZsh{} volt}
                                  \PY{n}{f} \PY{o}{=} \PY{n}{input\PYZus{}freq}\PY{o}{.}\PY{n}{f}\PY{p}{,}   \PY{c+c1}{\PYZsh{} Hz }
                                  \PY{n}{vin} \PY{o}{=} \PY{l+m+mi}{0}\PY{p}{,} 
                                  \PY{n}{R2} \PY{o}{=} \PY{l+m+mf}{1.58e3}\PY{p}{,}   \PY{c+c1}{\PYZsh{} ohm}
                                  \PY{n}{C2} \PY{o}{=} \PY{l+m+mf}{1e\PYZhy{}4}\PY{p}{,}  \PY{c+c1}{\PYZsh{} farad}
                                  \PY{n}{R3} \PY{o}{=} \PY{l+m+mi}{604}\PY{p}{,}   \PY{c+c1}{\PYZsh{} ohm}
                                  \PY{n}{C3} \PY{o}{=} \PY{l+m+mf}{1e\PYZhy{}4}\PY{p}{,}  \PY{c+c1}{\PYZsh{} farad}
                                  \PY{n}{R4} \PY{o}{=} \PY{l+m+mi}{604}\PY{p}{,}   \PY{c+c1}{\PYZsh{} ohm}
                                  \PY{n}{C4} \PY{o}{=} \PY{l+m+mf}{1e\PYZhy{}4}  \PY{c+c1}{\PYZsh{} farad}
                                 \PY{p}{)}
\end{Verbatim}


\begin{Verbatim}[commandchars=\\\{\}]
{\color{outcolor}Out[{\color{outcolor}45}]:} R1      604.0000
         C1        0.0001
         A       255.0000
         f       100.0000
         vin       0.0000
         R2     1580.0000
         C2        0.0001
         R3      604.0000
         C3        0.0001
         R4      604.0000
         C4        0.0001
         dtype: float64
\end{Verbatim}
            
    \begin{Verbatim}[commandchars=\\\{\}]
{\color{incolor}In [{\color{incolor}46}]:} \PY{n}{results5}\PY{p}{,} \PY{n}{results6}\PY{p}{,} \PY{n}{results7}\PY{p}{,} \PY{n}{results8}\PY{p}{,} \PY{n}{ratio} \PY{o}{=} \PY{n}{run\PYZus{}sim}\PY{p}{(}\PY{n}{input\PYZus{}freq}\PY{p}{,} 
                                                                 \PY{n}{make\PYZus{}system}\PY{p}{,} 
                                                                 \PY{n}{slope\PYZus{}func\PYZus{}init}\PY{p}{,} 
                                                                 \PY{n}{slope\PYZus{}func\PYZus{}high\PYZus{}pass}\PY{p}{,} 
                                                                 \PY{n}{slope\PYZus{}func\PYZus{}low\PYZus{}pass}\PY{p}{,} 
                                                                 \PY{n}{G1}\PY{p}{,} \PY{n}{G2}\PY{p}{,} \PY{n}{run\PYZus{}sim\PYZus{}system3}\PY{p}{)}
\end{Verbatim}


    \begin{Verbatim}[commandchars=\\\{\}]
{\color{incolor}In [{\color{incolor}47}]:} \PY{n}{plot\PYZus{}results}\PY{p}{(}\PY{n}{results5}\PY{p}{)}
         \PY{n}{plot\PYZus{}results}\PY{p}{(}\PY{n}{results6}\PY{p}{)}
         \PY{n}{plot\PYZus{}results}\PY{p}{(}\PY{n}{results7}\PY{p}{)}
         \PY{n}{plot\PYZus{}results}\PY{p}{(}\PY{n}{results8}\PY{p}{)}
\end{Verbatim}


    \begin{center}
    \adjustimage{max size={0.9\linewidth}{0.9\paperheight}}{output_92_0.png}
    \end{center}
    { \hspace*{\fill} \\}
    
    \begin{Verbatim}[commandchars=\\\{\}]
{\color{incolor}In [{\color{incolor}48}]:} \PY{n}{sweep3} \PY{o}{=} \PY{n}{SweepSeries}\PY{p}{(}\PY{p}{)}
         \PY{k}{for} \PY{n}{f} \PY{o+ow}{in} \PY{n}{linspace}\PY{p}{(}\PY{l+m+mi}{1}\PY{p}{,} \PY{l+m+mi}{100}\PY{p}{,} \PY{l+m+mi}{50}\PY{p}{)}\PY{p}{:}
             \PY{n}{input\PYZus{}freq}\PY{o}{.}\PY{n}{f}\PY{o}{=}\PY{n}{f}
             \PY{n}{state} \PY{o}{=} \PY{n}{run\PYZus{}sim}\PY{p}{(}\PY{n}{input\PYZus{}freq}\PY{p}{,} \PY{n}{make\PYZus{}system}\PY{p}{,} 
                             \PY{n}{slope\PYZus{}func\PYZus{}init}\PY{p}{,} 
                             \PY{n}{slope\PYZus{}func\PYZus{}high\PYZus{}pass}\PY{p}{,} 
                             \PY{n}{slope\PYZus{}func\PYZus{}low\PYZus{}pass}\PY{p}{,} 
                             \PY{n}{G1}\PY{p}{,} \PY{n}{G2}\PY{p}{,} \PY{n}{run\PYZus{}sim\PYZus{}system3}\PY{p}{)}\PY{o}{.}\PY{n}{ratio}
             \PY{n}{sweep3}\PY{p}{[}\PY{n}{f}\PY{p}{]} \PY{o}{=} \PY{n}{state}\PY{p}{;}
\end{Verbatim}


    These are the third set of R and C values we tested.

    \begin{Verbatim}[commandchars=\\\{\}]
{\color{incolor}In [{\color{incolor}49}]:} \PY{n}{run\PYZus{}sim\PYZus{}system4} \PY{o}{=} \PY{n}{System}\PY{p}{(}\PY{n}{R1} \PY{o}{=} \PY{l+m+mf}{4.99e3}\PY{p}{,} \PY{c+c1}{\PYZsh{} ohm}
                                  \PY{n}{C1} \PY{o}{=} \PY{l+m+mf}{1e\PYZhy{}6}\PY{p}{,}  \PY{c+c1}{\PYZsh{} farad }
                                  \PY{n}{A} \PY{o}{=} \PY{l+m+mi}{5} \PY{o}{*} \PY{n}{G1}\PY{p}{,} \PY{c+c1}{\PYZsh{} volt}
                                  \PY{n}{f} \PY{o}{=} \PY{n}{input\PYZus{}freq}\PY{o}{.}\PY{n}{f}\PY{p}{,}   \PY{c+c1}{\PYZsh{} Hz }
                                  \PY{n}{vin} \PY{o}{=} \PY{l+m+mi}{0}\PY{p}{,} 
                                  \PY{n}{R2} \PY{o}{=} \PY{l+m+mf}{1.58e3}\PY{p}{,}   \PY{c+c1}{\PYZsh{} ohm}
                                  \PY{n}{C2} \PY{o}{=} \PY{l+m+mf}{1e\PYZhy{}4}\PY{p}{,}  \PY{c+c1}{\PYZsh{} farad}
                                  \PY{n}{R3} \PY{o}{=} \PY{l+m+mf}{2.00e4}\PY{p}{,}   \PY{c+c1}{\PYZsh{} ohm}
                                  \PY{n}{C3} \PY{o}{=} \PY{l+m+mf}{1e\PYZhy{}6}\PY{p}{,}  \PY{c+c1}{\PYZsh{} farad}
                                  \PY{n}{R4} \PY{o}{=} \PY{l+m+mf}{4.9e4}\PY{p}{,}   \PY{c+c1}{\PYZsh{} ohm}
                                  \PY{n}{C4} \PY{o}{=} \PY{l+m+mf}{1e\PYZhy{}6}  \PY{c+c1}{\PYZsh{} farad}
                                 \PY{p}{)}
\end{Verbatim}


\begin{Verbatim}[commandchars=\\\{\}]
{\color{outcolor}Out[{\color{outcolor}49}]:} R1      4990.000000
         C1         0.000001
         A        255.000000
         f        100.000000
         vin        0.000000
         R2      1580.000000
         C2         0.000100
         R3     20000.000000
         C3         0.000001
         R4     49000.000000
         C4         0.000001
         dtype: float64
\end{Verbatim}
            
    \begin{Verbatim}[commandchars=\\\{\}]
{\color{incolor}In [{\color{incolor}50}]:} \PY{n}{results5}\PY{p}{,} \PY{n}{results6}\PY{p}{,} \PY{n}{results7}\PY{p}{,} \PY{n}{results8}\PY{p}{,} \PY{n}{ratio} \PY{o}{=} \PY{n}{run\PYZus{}sim}\PY{p}{(}\PY{n}{input\PYZus{}freq}\PY{p}{,} 
                                                                 \PY{n}{make\PYZus{}system}\PY{p}{,} 
                                                                 \PY{n}{slope\PYZus{}func\PYZus{}init}\PY{p}{,} 
                                                                 \PY{n}{slope\PYZus{}func\PYZus{}high\PYZus{}pass}\PY{p}{,} 
                                                                 \PY{n}{slope\PYZus{}func\PYZus{}low\PYZus{}pass}\PY{p}{,} 
                                                                 \PY{n}{G1}\PY{p}{,} \PY{n}{G2}\PY{p}{,} \PY{n}{run\PYZus{}sim\PYZus{}system4}\PY{p}{)}
\end{Verbatim}


    \begin{Verbatim}[commandchars=\\\{\}]
{\color{incolor}In [{\color{incolor}51}]:} \PY{n}{plot\PYZus{}results}\PY{p}{(}\PY{n}{results5}\PY{p}{)}
         \PY{n}{plot\PYZus{}results}\PY{p}{(}\PY{n}{results6}\PY{p}{)}
         \PY{n}{plot\PYZus{}results}\PY{p}{(}\PY{n}{results7}\PY{p}{)}
         \PY{n}{plot\PYZus{}results}\PY{p}{(}\PY{n}{results8}\PY{p}{)}
\end{Verbatim}


    \begin{center}
    \adjustimage{max size={0.9\linewidth}{0.9\paperheight}}{output_97_0.png}
    \end{center}
    { \hspace*{\fill} \\}
    
    \begin{Verbatim}[commandchars=\\\{\}]
{\color{incolor}In [{\color{incolor}52}]:} \PY{n}{sweep4} \PY{o}{=} \PY{n}{SweepSeries}\PY{p}{(}\PY{p}{)}
         \PY{k}{for} \PY{n}{f} \PY{o+ow}{in} \PY{n}{linspace}\PY{p}{(}\PY{l+m+mi}{1}\PY{p}{,} \PY{l+m+mi}{100}\PY{p}{,} \PY{l+m+mi}{50}\PY{p}{)}\PY{p}{:}
             \PY{n}{input\PYZus{}freq}\PY{o}{.}\PY{n}{f}\PY{o}{=}\PY{n}{f}
             \PY{n}{state} \PY{o}{=} \PY{n}{run\PYZus{}sim}\PY{p}{(}\PY{n}{input\PYZus{}freq}\PY{p}{,} \PY{n}{make\PYZus{}system}\PY{p}{,} 
                             \PY{n}{slope\PYZus{}func\PYZus{}init}\PY{p}{,} 
                             \PY{n}{slope\PYZus{}func\PYZus{}high\PYZus{}pass}\PY{p}{,} 
                             \PY{n}{slope\PYZus{}func\PYZus{}low\PYZus{}pass}\PY{p}{,} 
                             \PY{n}{G1}\PY{p}{,} \PY{n}{G2}\PY{p}{,} \PY{n}{run\PYZus{}sim\PYZus{}system4}\PY{p}{)}\PY{o}{.}\PY{n}{ratio}
             \PY{n}{sweep4}\PY{p}{[}\PY{n}{f}\PY{p}{]} \PY{o}{=} \PY{n}{state}\PY{p}{;}
\end{Verbatim}


    These are the fourth set of R and C values we tested.

    \begin{Verbatim}[commandchars=\\\{\}]
{\color{incolor}In [{\color{incolor}53}]:} \PY{n}{run\PYZus{}sim\PYZus{}system5} \PY{o}{=} \PY{n}{System}\PY{p}{(}\PY{n}{R1} \PY{o}{=} \PY{l+m+mf}{2e3}\PY{p}{,} \PY{c+c1}{\PYZsh{} ohm}
                                  \PY{n}{C1} \PY{o}{=} \PY{l+m+mf}{1e\PYZhy{}5}\PY{p}{,}  \PY{c+c1}{\PYZsh{} farad }
                                  \PY{n}{A} \PY{o}{=} \PY{l+m+mi}{5} \PY{o}{*} \PY{n}{G1}\PY{p}{,} \PY{c+c1}{\PYZsh{} volt}
                                  \PY{n}{f} \PY{o}{=} \PY{n}{input\PYZus{}freq}\PY{o}{.}\PY{n}{f}\PY{p}{,}   \PY{c+c1}{\PYZsh{} Hz }
                                  \PY{n}{vin} \PY{o}{=} \PY{l+m+mi}{0}\PY{p}{,} 
                                  \PY{n}{R2} \PY{o}{=} \PY{l+m+mf}{1.58e3}\PY{p}{,}   \PY{c+c1}{\PYZsh{} ohm}
                                  \PY{n}{C2} \PY{o}{=} \PY{l+m+mf}{1e\PYZhy{}5}\PY{p}{,}  \PY{c+c1}{\PYZsh{} farad}
                                  \PY{n}{R3} \PY{o}{=} \PY{l+m+mf}{2.00e2}\PY{p}{,}   \PY{c+c1}{\PYZsh{} ohm}
                                  \PY{n}{C3} \PY{o}{=} \PY{l+m+mf}{1e\PYZhy{}4}\PY{p}{,}  \PY{c+c1}{\PYZsh{} farad}
                                  \PY{n}{R4} \PY{o}{=} \PY{l+m+mf}{4.99e3}\PY{p}{,}   \PY{c+c1}{\PYZsh{} ohm}
                                  \PY{n}{C4} \PY{o}{=} \PY{l+m+mf}{1e\PYZhy{}6}  \PY{c+c1}{\PYZsh{} farad}
                                 \PY{p}{)}
\end{Verbatim}


\begin{Verbatim}[commandchars=\\\{\}]
{\color{outcolor}Out[{\color{outcolor}53}]:} R1     2000.000000
         C1        0.000010
         A       255.000000
         f       100.000000
         vin       0.000000
         R2     1580.000000
         C2        0.000010
         R3      200.000000
         C3        0.000100
         R4     4990.000000
         C4        0.000001
         dtype: float64
\end{Verbatim}
            
    \begin{Verbatim}[commandchars=\\\{\}]
{\color{incolor}In [{\color{incolor}54}]:} \PY{n}{results5}\PY{p}{,} \PY{n}{results6}\PY{p}{,} \PY{n}{results7}\PY{p}{,} \PY{n}{results8}\PY{p}{,} \PY{n}{ratio} \PY{o}{=} \PY{n}{run\PYZus{}sim}\PY{p}{(}\PY{n}{input\PYZus{}freq}\PY{p}{,} 
                                                                 \PY{n}{make\PYZus{}system}\PY{p}{,} 
                                                                 \PY{n}{slope\PYZus{}func\PYZus{}init}\PY{p}{,} 
                                                                 \PY{n}{slope\PYZus{}func\PYZus{}high\PYZus{}pass}\PY{p}{,} 
                                                                 \PY{n}{slope\PYZus{}func\PYZus{}low\PYZus{}pass}\PY{p}{,} 
                                                                 \PY{n}{G1}\PY{p}{,} \PY{n}{G2}\PY{p}{,} \PY{n}{run\PYZus{}sim\PYZus{}system5}\PY{p}{)}
\end{Verbatim}


    \begin{Verbatim}[commandchars=\\\{\}]
{\color{incolor}In [{\color{incolor}55}]:} \PY{n}{plot\PYZus{}results}\PY{p}{(}\PY{n}{results5}\PY{p}{)}
         \PY{n}{plot\PYZus{}results}\PY{p}{(}\PY{n}{results6}\PY{p}{)}
         \PY{n}{plot\PYZus{}results}\PY{p}{(}\PY{n}{results7}\PY{p}{)}
         \PY{n}{plot\PYZus{}results}\PY{p}{(}\PY{n}{results8}\PY{p}{)}
\end{Verbatim}


    \begin{center}
    \adjustimage{max size={0.9\linewidth}{0.9\paperheight}}{output_102_0.png}
    \end{center}
    { \hspace*{\fill} \\}
    
    \begin{Verbatim}[commandchars=\\\{\}]
{\color{incolor}In [{\color{incolor}56}]:} \PY{n}{sweep5} \PY{o}{=} \PY{n}{SweepSeries}\PY{p}{(}\PY{p}{)}
         \PY{k}{for} \PY{n}{f} \PY{o+ow}{in} \PY{n}{linspace}\PY{p}{(}\PY{l+m+mi}{1}\PY{p}{,} \PY{l+m+mi}{100}\PY{p}{,} \PY{l+m+mi}{50}\PY{p}{)}\PY{p}{:}
             \PY{n}{input\PYZus{}freq}\PY{o}{.}\PY{n}{f}\PY{o}{=}\PY{n}{f}
             \PY{n}{state} \PY{o}{=} \PY{n}{run\PYZus{}sim}\PY{p}{(}\PY{n}{input\PYZus{}freq}\PY{p}{,} \PY{n}{make\PYZus{}system}\PY{p}{,} 
                             \PY{n}{slope\PYZus{}func\PYZus{}init}\PY{p}{,} 
                             \PY{n}{slope\PYZus{}func\PYZus{}high\PYZus{}pass}\PY{p}{,} 
                             \PY{n}{slope\PYZus{}func\PYZus{}low\PYZus{}pass}\PY{p}{,} 
                             \PY{n}{G1}\PY{p}{,} \PY{n}{G2}\PY{p}{,} \PY{n}{run\PYZus{}sim\PYZus{}system5}\PY{p}{)}\PY{o}{.}\PY{n}{ratio}
             \PY{n}{sweep5}\PY{p}{[}\PY{n}{f}\PY{p}{]} \PY{o}{=} \PY{n}{state}\PY{p}{;}
\end{Verbatim}


    These are the fifth set of R and C values we tested.

    \begin{Verbatim}[commandchars=\\\{\}]
{\color{incolor}In [{\color{incolor}57}]:} \PY{n}{run\PYZus{}sim\PYZus{}system6} \PY{o}{=} \PY{n}{System}\PY{p}{(}\PY{n}{R1} \PY{o}{=} \PY{l+m+mf}{2e3}\PY{p}{,} \PY{c+c1}{\PYZsh{} ohm}
                                  \PY{n}{C1} \PY{o}{=} \PY{l+m+mf}{1e\PYZhy{}5}\PY{p}{,}  \PY{c+c1}{\PYZsh{} farad }
                                  \PY{n}{A} \PY{o}{=} \PY{l+m+mi}{5} \PY{o}{*} \PY{n}{G1}\PY{p}{,} \PY{c+c1}{\PYZsh{} volt}
                                  \PY{n}{f} \PY{o}{=} \PY{n}{input\PYZus{}freq}\PY{o}{.}\PY{n}{f}\PY{p}{,}   \PY{c+c1}{\PYZsh{} Hz }
                                  \PY{n}{vin} \PY{o}{=} \PY{l+m+mi}{0}\PY{p}{,} 
                                  \PY{n}{R2} \PY{o}{=} \PY{l+m+mf}{1.58e3}\PY{p}{,}   \PY{c+c1}{\PYZsh{} ohm}
                                  \PY{n}{C2} \PY{o}{=} \PY{l+m+mf}{1e\PYZhy{}5}\PY{p}{,}  \PY{c+c1}{\PYZsh{} farad}
                                  \PY{n}{R3} \PY{o}{=} \PY{l+m+mf}{2.00e2}\PY{p}{,}   \PY{c+c1}{\PYZsh{} ohm}
                                  \PY{n}{C3} \PY{o}{=} \PY{l+m+mf}{1e\PYZhy{}4}\PY{p}{,}  \PY{c+c1}{\PYZsh{} farad}
                                  \PY{n}{R4} \PY{o}{=} \PY{l+m+mf}{4.99e4}\PY{p}{,}   \PY{c+c1}{\PYZsh{} ohm}
                                  \PY{n}{C4} \PY{o}{=} \PY{l+m+mf}{1e\PYZhy{}6}  \PY{c+c1}{\PYZsh{} farad}
                                 \PY{p}{)}
\end{Verbatim}


\begin{Verbatim}[commandchars=\\\{\}]
{\color{outcolor}Out[{\color{outcolor}57}]:} R1      2000.000000
         C1         0.000010
         A        255.000000
         f        100.000000
         vin        0.000000
         R2      1580.000000
         C2         0.000010
         R3       200.000000
         C3         0.000100
         R4     49900.000000
         C4         0.000001
         dtype: float64
\end{Verbatim}
            
    \begin{Verbatim}[commandchars=\\\{\}]
{\color{incolor}In [{\color{incolor}58}]:} \PY{n}{results5}\PY{p}{,} \PY{n}{results6}\PY{p}{,} \PY{n}{results7}\PY{p}{,} \PY{n}{results8}\PY{p}{,} \PY{n}{ratio} \PY{o}{=} \PY{n}{run\PYZus{}sim}\PY{p}{(}\PY{n}{input\PYZus{}freq}\PY{p}{,} 
                                                                 \PY{n}{make\PYZus{}system}\PY{p}{,} 
                                                                 \PY{n}{slope\PYZus{}func\PYZus{}init}\PY{p}{,} 
                                                                 \PY{n}{slope\PYZus{}func\PYZus{}high\PYZus{}pass}\PY{p}{,} 
                                                                 \PY{n}{slope\PYZus{}func\PYZus{}low\PYZus{}pass}\PY{p}{,} 
                                                                 \PY{n}{G1}\PY{p}{,} \PY{n}{G2}\PY{p}{,} \PY{n}{run\PYZus{}sim\PYZus{}system6}\PY{p}{)}
\end{Verbatim}


    \begin{Verbatim}[commandchars=\\\{\}]
{\color{incolor}In [{\color{incolor}59}]:} \PY{n}{plot\PYZus{}results}\PY{p}{(}\PY{n}{results5}\PY{p}{)}
         \PY{n}{plot\PYZus{}results}\PY{p}{(}\PY{n}{results6}\PY{p}{)}
         \PY{n}{plot\PYZus{}results}\PY{p}{(}\PY{n}{results7}\PY{p}{)}
         \PY{n}{plot\PYZus{}results}\PY{p}{(}\PY{n}{results8}\PY{p}{)}
\end{Verbatim}


    \begin{center}
    \adjustimage{max size={0.9\linewidth}{0.9\paperheight}}{output_107_0.png}
    \end{center}
    { \hspace*{\fill} \\}
    
    \begin{Verbatim}[commandchars=\\\{\}]
{\color{incolor}In [{\color{incolor}60}]:} \PY{n}{sweep6} \PY{o}{=} \PY{n}{SweepSeries}\PY{p}{(}\PY{p}{)}
         \PY{k}{for} \PY{n}{f} \PY{o+ow}{in} \PY{n}{linspace}\PY{p}{(}\PY{l+m+mi}{1}\PY{p}{,} \PY{l+m+mi}{100}\PY{p}{,} \PY{l+m+mi}{50}\PY{p}{)}\PY{p}{:}
             \PY{n}{input\PYZus{}freq}\PY{o}{.}\PY{n}{f}\PY{o}{=}\PY{n}{f}
             \PY{n}{state} \PY{o}{=} \PY{n}{run\PYZus{}sim}\PY{p}{(}\PY{n}{input\PYZus{}freq}\PY{p}{,} \PY{n}{make\PYZus{}system}\PY{p}{,} 
                             \PY{n}{slope\PYZus{}func\PYZus{}init}\PY{p}{,} 
                             \PY{n}{slope\PYZus{}func\PYZus{}high\PYZus{}pass}\PY{p}{,} 
                             \PY{n}{slope\PYZus{}func\PYZus{}low\PYZus{}pass}\PY{p}{,} 
                             \PY{n}{G1}\PY{p}{,} \PY{n}{G2}\PY{p}{,} \PY{n}{run\PYZus{}sim\PYZus{}system6}\PY{p}{)}\PY{o}{.}\PY{n}{ratio}
             \PY{n}{sweep6}\PY{p}{[}\PY{n}{f}\PY{p}{]} \PY{o}{=} \PY{n}{state}\PY{p}{;}
\end{Verbatim}


    \subsubsection{Results 3:}\label{results-3}

    With the changes in R and C values, the input signal will attenuate
differently. We then passed various frequencies through the modeled
circuit to create a Bode Plot. A comparison of the Bode plots are shown
below.

    \begin{Verbatim}[commandchars=\\\{\}]
{\color{incolor}In [{\color{incolor}61}]:} \PY{n}{plot}\PY{p}{(}\PY{n}{sweep}\PY{p}{,}  \PY{n}{label} \PY{o}{=} \PY{l+s+s1}{\PYZsq{}}\PY{l+s+s1}{Model 1 Data}\PY{l+s+s1}{\PYZsq{}}\PY{p}{)}
         \PY{n}{plot}\PY{p}{(}\PY{n}{sweep2}\PY{p}{,} \PY{n}{label} \PY{o}{=} \PY{l+s+s1}{\PYZsq{}}\PY{l+s+s1}{Adjusted RC Values 1}\PY{l+s+s1}{\PYZsq{}}\PY{p}{)}
         \PY{n}{plot}\PY{p}{(}\PY{n}{sweep3}\PY{p}{,} \PY{n}{label} \PY{o}{=} \PY{l+s+s1}{\PYZsq{}}\PY{l+s+s1}{Adjusted RC Values 2}\PY{l+s+s1}{\PYZsq{}}\PY{p}{)}
         \PY{n}{plot}\PY{p}{(}\PY{n}{sweep4}\PY{p}{,} \PY{n}{label} \PY{o}{=} \PY{l+s+s1}{\PYZsq{}}\PY{l+s+s1}{Adjusted RC Values 3}\PY{l+s+s1}{\PYZsq{}}\PY{p}{)}
         \PY{n}{plot}\PY{p}{(}\PY{n}{sweep5}\PY{p}{,} \PY{n}{label} \PY{o}{=} \PY{l+s+s1}{\PYZsq{}}\PY{l+s+s1}{Adjusted RC Values 4}\PY{l+s+s1}{\PYZsq{}}\PY{p}{)}
         \PY{n}{plot}\PY{p}{(}\PY{n}{sweep6}\PY{p}{,} \PY{n}{label} \PY{o}{=} \PY{l+s+s1}{\PYZsq{}}\PY{l+s+s1}{Adjusted RC Values 5}\PY{l+s+s1}{\PYZsq{}}\PY{p}{)}
         
         \PY{n}{decorate}\PY{p}{(}\PY{n}{xlabel}\PY{o}{=}\PY{l+s+s1}{\PYZsq{}}\PY{l+s+s1}{Frequency (Hz)}\PY{l+s+s1}{\PYZsq{}}\PY{p}{,}
                  \PY{n}{ylabel}\PY{o}{=}\PY{l+s+s1}{\PYZsq{}}\PY{l+s+s1}{Amplitude}\PY{l+s+s1}{\PYZsq{}}\PY{p}{,}
                  \PY{n}{xscale}\PY{o}{=}\PY{l+s+s1}{\PYZsq{}}\PY{l+s+s1}{log}\PY{l+s+s1}{\PYZsq{}}\PY{p}{,}
                  \PY{n}{yscale}\PY{o}{=}\PY{l+s+s1}{\PYZsq{}}\PY{l+s+s1}{log}\PY{l+s+s1}{\PYZsq{}}\PY{p}{,}
                  \PY{n}{title}\PY{o}{=}\PY{l+s+s1}{\PYZsq{}}\PY{l+s+s1}{Bode Plot Comparison}\PY{l+s+s1}{\PYZsq{}}\PY{p}{)}
\end{Verbatim}


    \begin{center}
    \adjustimage{max size={0.9\linewidth}{0.9\paperheight}}{output_111_0.png}
    \end{center}
    { \hspace*{\fill} \\}
    
    \subsubsection{Interpretation:}\label{interpretation}

    The response of the circuit to different cutoff frequencies can be seen
in the Bode plots above. As the cutoff frequency of the high pass filter
increases, the Bode Plot line becomes more shallow, and as the cutoff
frequency of the low pass filters increases, the y-intercept of the Bode
Plot decreases.

    A few of the drawbacks with our model are that the high pass slope
function can only be used if there is only one high pass filter and it
is comes after the first low pass filter, we do not model electrical
interference, and we use a cosine wave to model the input signal. We
know that a heartbeat is not a cosine wave, and thus, we are unsure as
to whether a heart signal would work with this model.

    We wrote various functions in attempts to best fit our model, and stuck
with the ones that worked. We also varied RC and frequency values to
determine if our model was functioning the way we anticipated.

    \subsubsection{Abstract:}\label{abstract}

    The question we addressed was, how does this circuit respond with
changes in RC values? The response of the circuit to different cutoff
frequencies can be seen in the Bode plots above. As the cutoff frequency
of the high pass filter increases, the Bode Plot line becomes more
shallow, and as the cutoff frequency of the low pass filters increases,
the y-intercept of the Bode Plot decreases. Using the initial circuit,
we can see that the output signal will always be attenuated; however,
the extent of the attenuation depends on frequency and the cutoff
frequencies of the filters.

    \begin{Verbatim}[commandchars=\\\{\}]
{\color{incolor}In [{\color{incolor}62}]:} \PY{n}{plot}\PY{p}{(}\PY{n}{sweep}\PY{p}{,}  \PY{n}{label} \PY{o}{=} \PY{l+s+s1}{\PYZsq{}}\PY{l+s+s1}{Model 1 Data}\PY{l+s+s1}{\PYZsq{}}\PY{p}{)}
         \PY{n}{plot}\PY{p}{(}\PY{n}{sweep2}\PY{p}{,} \PY{n}{label} \PY{o}{=} \PY{l+s+s1}{\PYZsq{}}\PY{l+s+s1}{Adjusted RC Values 1}\PY{l+s+s1}{\PYZsq{}}\PY{p}{)}
         \PY{n}{plot}\PY{p}{(}\PY{n}{sweep3}\PY{p}{,} \PY{n}{label} \PY{o}{=} \PY{l+s+s1}{\PYZsq{}}\PY{l+s+s1}{Adjusted RC Values 2}\PY{l+s+s1}{\PYZsq{}}\PY{p}{)}
         \PY{n}{plot}\PY{p}{(}\PY{n}{sweep4}\PY{p}{,} \PY{n}{label} \PY{o}{=} \PY{l+s+s1}{\PYZsq{}}\PY{l+s+s1}{Adjusted RC Values 3}\PY{l+s+s1}{\PYZsq{}}\PY{p}{)}
         \PY{n}{plot}\PY{p}{(}\PY{n}{sweep5}\PY{p}{,} \PY{n}{label} \PY{o}{=} \PY{l+s+s1}{\PYZsq{}}\PY{l+s+s1}{Adjusted RC Values 4}\PY{l+s+s1}{\PYZsq{}}\PY{p}{)}
         \PY{n}{plot}\PY{p}{(}\PY{n}{sweep6}\PY{p}{,} \PY{n}{label} \PY{o}{=} \PY{l+s+s1}{\PYZsq{}}\PY{l+s+s1}{Adjusted RC Values 5}\PY{l+s+s1}{\PYZsq{}}\PY{p}{)}
         
         \PY{n}{decorate}\PY{p}{(}\PY{n}{xlabel}\PY{o}{=}\PY{l+s+s1}{\PYZsq{}}\PY{l+s+s1}{Frequency (Hz)}\PY{l+s+s1}{\PYZsq{}}\PY{p}{,}
                  \PY{n}{ylabel}\PY{o}{=}\PY{l+s+s1}{\PYZsq{}}\PY{l+s+s1}{Amplitude}\PY{l+s+s1}{\PYZsq{}}\PY{p}{,}
                  \PY{n}{xscale}\PY{o}{=}\PY{l+s+s1}{\PYZsq{}}\PY{l+s+s1}{log}\PY{l+s+s1}{\PYZsq{}}\PY{p}{,}
                  \PY{n}{yscale}\PY{o}{=}\PY{l+s+s1}{\PYZsq{}}\PY{l+s+s1}{log}\PY{l+s+s1}{\PYZsq{}}\PY{p}{,}
                  \PY{n}{title}\PY{o}{=}\PY{l+s+s1}{\PYZsq{}}\PY{l+s+s1}{Bode Plot Comparison}\PY{l+s+s1}{\PYZsq{}}\PY{p}{)}
\end{Verbatim}


    \begin{center}
    \adjustimage{max size={0.9\linewidth}{0.9\paperheight}}{output_118_0.png}
    \end{center}
    { \hspace*{\fill} \\}
    

    % Add a bibliography block to the postdoc
    
    
    
    \end{document}
